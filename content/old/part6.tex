
\section{cf's 1303B problem}\timetx{2020-02-14 16:35:21.606044}

首先因为\vb|scanf|和\vb|long long|之间,
不太熟悉而所以处处碰壁。

\subsection{关于\vb|long long|和\vb|scanf|}
测试代码如下:
\begin{lstlisting}[language=C++]
#include<bits/stdc++.h>
using namespace std;
using ll = long long;

int main(){
    ll a, c;
    int b;
    scanf("%d%d%lld", &a, &b, &c);
    cout << a << ' ' << b << ' ' << c << endl;
    printf("%lld %d %d\n", a, b, c);
    return 0;
}
\end{lstlisting}

测试输入输出如下:
\begin{lstlisting}[mathescape = false]
$ ./a.out
4294967296 4294967296 4294967296
0 0 4294967296
0 0 0
\end{lstlisting}

可以看到哇,对于\vb|a|来说,只有使用\vb|\%lld|,
才会使得它把正确的数据储存到相关的内存位置上,
而不然的话,它则会使用错误的数据,
可以看到高位的1被舍弃掉了。

同理,\vb|printf|则根本没有读高位的1。

\subsection{其他人的答案}
在这道题中我需要求到$\lceil\frac{n}{m}\rceil$,
而众所周知C++中整数相除是得不到小数的,所以我用了:
\begin{lstlisting}[language=C++]
(n % m)?n / m + 1: n / m
\end{lstlisting}

但换一种思考方式,只有$m|n$的时候,
$n($ mod $m)$才会等于$0$,$\lfloor\frac{n}{m}\rfloor$,
在这种时候可以直接等于$\lceil\frac{n}{m}\rceil$,
但是在其他时候都有:
$$\lfloor\frac{n}{m}\rfloor + 1 = \lceil\frac{n}{m}\rceil$$
所以说只要令$n$加上$m-1$成为$n'$时,
当原$n$存在有:$n($ mod $m) = 0$时,
变化后的$n'$有$\lfloor\frac{n'}{m}\rfloor$的值不变。
但是对于原$n$有$n($ mod $m)\in [1, m-1]$时,
都有变化后恒成立:$\lfloor\frac{n'}{m}\rfloor$有
原$n$的$\lfloor\frac{n}{m}\rfloor + 1$的值。

故上代码的等价代码为:
\begin{lstlisting}[language=C++]
(n + m - 1)/ m
\end{lstlisting}


\section{cf's 1303C problem}\timetx{2020-02-14 19:14:25.856520}

这道题,给定了一个密码,
密码的每两个相邻的字母在键盘上也是相邻的,
求取符合条件的一维键盘布局。

很明显是要求输入一个密码,输出一个布局格式。
而对于格式而言,因为是一维的,
所以我是想着可以根据这个建一个图,
其中一般情况下,每个点要不然有2个边,
要不然没有边(就是没有构成密码的字母),
而有一个边的点就是密码区块的开始和结束,
所有的密码都是在密码区块中的。

以上是我的思路。
当然具体要复杂一些,包括判断有无环,
还有密码长度为一的特殊情况。

\subsection{其他人的答案}
有人是这么做的:
使用\vb|bitset|记录所有密码的字符集,
如果密码的下一位没有在字符集中,
则把字符链接到密码区块的前面中。
如果一直是这种线性的结构的话,
维持\vb|p = 0|的位置,\vb|p|一直指着
最新的字符。

这样子的话,到时候就能直接输出密码区块,
之后再把不在字符集中的所有字符输出。

但是密码的下一位出现在了字符集中,
而且\vb|p = 0|(这代表了它之前是线性的),
而且最重要的是,
这个字符不在正在增长的密码区块\vb|result|的
第\vb|p+1|位中,
也就是和本位\vb|result[p]|的上一位不一样,
说明了下一位在密码区块中已经有左右两个字母,
还要和本位在一起,这在一维上是矛盾的。

但是\vb|p|不可能一直指向最新的元素,
如果它指向最老的元素,即\vb|p = a.size() - 1|,
此时有一个不在字符集中的元素,这可以把该元素
链接在后面以成为最老的元素
(因为这个元素和之前那个元素很近,
又不在密码区块中,那只能在密码区块的另一边中了啦)。
但是在的话,而且\vb|result[p-1]=ch|,
就令\vb|p--|,让它指向它应该在的位置。

以下是最主要的循环:
\begin{lstlisting}[language=C++]
for(auto& ch: input){
    // if ch is in alphabet:
    if(alphabet[ch-'a'])
        // if it is not at the tail, so we can move p
        if(p < result.size() - 1 && result[p+1] == ch)
            p++;
        // if it is not at the head, so we can move p
        else if(p > 0 && result[p-1] == ch)
            p--;
        // oh we can't move it!
        else{
            flag = true;
            break;
        }
    // if ch is not in alphabet, 
    // then should link it to result,
    // its head or tail, whatever.
    else
        // if it is on the head - it is easy
        if(p == 0)
            result = ch + result;
        // or it is on the tail
        else if(p == result.size() - 1){
            result += ch;
            p++;
        // it is on the inside of result
        } else {
            flag = true;
            break;
        }
    alphabet.set(ch - 'a');
}
\end{lstlisting}


\section{cf's 1304C problem}\timetx{2020-02-16 10:11:01.261449}

这道题需要判断有无相对应的回文串,
所以我必须要得到一个字符串的\vb|reversed|版本。

在我的程序中使用的是自己实现的\vb|reversed|函数,
但是在标准库中已经有了\vb|std::reverse|函数,
传入两个迭代器指向首和末,实现原址颠倒。
里面用到的函数有\vb|std::iter_swap|

此外我之前企图在迭代的过程中更改原容器的值。
但是更改容器的操作好像(应该)会引发一些错误。

此外,我试图把输入和运行相隔开,但是这样子的话,
可能甚至会让情况复杂化!!!
所以如何取舍呢,有点难欸。


\section{cf's 1304 problem}\timetx{2020-02-16 10:58:40.308151}

是时候学点新知识了,铛铛!\vb|struct|!
之前忙着做题,不敢使用这个怕做错了就麻烦了,
结果我还是完全记得这个语法结构的说。

使用结构肯定比使用数组要好,
毕竟数组是使用来储存意义相同的数据,
而结构是用来储存意义不一样的结构。
用来比较的话,一个相当于Python中的列表,
另一个就像元组和类。

此外,鉴于我经常在里面犯一些低级错误,
重载输出操作符也是很有必要的,比如说:
\begin{lstlisting}[language=C++]
ostream& operator<<(ostream& os, const vector<p>& it){
    os << '[';
    for(auto &i: it){
        os << "{" << i.t << ", " << i.h << ", " << i.l << '}, ';
    os << ']' << endl;
}
\end{lstlisting}

其他的和紫名大佬差不多,除了我把循环和处理和输入输出都分开了。

我在想是不是没有必要再竞赛中把输入输出和处理隔开。

