
\section{cf's 1304 problem}\timetx{2020-02-16 10:58:40.308151}

是时候学点新知识了,铛铛!\vb|struct|!
之前忙着做题,不敢使用这个怕做错了就麻烦了,
结果我还是完全记得这个语法结构的说。

使用结构肯定比使用数组要好,
毕竟数组是使用来储存意义相同的数据,
而结构是用来储存意义不一样的结构。
用来比较的话,一个相当于Python中的列表,
另一个就像元组和类。

此外,鉴于我经常在里面犯一些低级错误,
重载输出操作符也是很有必要的,比如说:
\begin{lstlisting}[language=C++]
ostream& operator<<(ostream& os, const vector<p>& it){
    os << '[';
    for(auto &i: it){
        os << "{" << i.t << ", " << i.h << ", " << i.l << '}, ';
    os << ']' << endl;
}
\end{lstlisting}

其他的和紫名大佬差不多,除了我把循环和处理和输入输出都分开了。

我在想是不是没有必要再竞赛中把输入输出和处理隔开。


\section{cf's 1296C problem}\timetx{2020-02-16 22:15:33.313424}

我又被卡住啦!鼓掌!(啪啦啪啦)

这道题看起来很像是dp的题嘛,
但是没有想到不行欸,也是哈,毕竟是$O(n^2)$嘛。

这道题意思如下:有一个字符串,
要寻找它的最短的符合规则的连续子字符串,
而所谓规则,
就是该字符串中的\vb|'U'|和\vb|'D'|,
\vb|'R'|和\vb|'L'|的数量两两相等。

我是这么想的:
定义一个结构体$y = f(str)$来存储$str[i..j]$的四个值,
于是结构体满足:
$y_{i..j}=f(str[i..j])=f(str[i..j-1]+str[j])=
y_{i..j-1}+f(str[i])$

虽然我是菜鸡,但是动态规划什么的,
自底向上简简单单的啦。
说着我啪嗒啪嗒敲好了代码,
还为结构体重载了运算符。

结果\vb|time limit|,也是毕竟是$O(n^2)$的算法嘛,
暴力也是$O(n^2)$的复杂度对吧。

但是真的有比这个好的算法吗\ldots\ldots

\subsection{其他人的答案}
还是有更好的方法,这道题的目的就是为了找到$i$和$j$,
但是并没有必要找到$1\le i\le j\le len$中%
所有满足的$a[i..j]$的数据结构,因为这样子,
时间复杂度为$O(len^2)$。

但是从另外一种方向考虑的话,
这种题目的数据结构是有它的特点的:
对于$a[1..i]$和$a[1..j]$来说的话,
如果增加的\vb|'U'|和\vb|'D'|,
增加的\vb|'R'|和\vb|'L'|两两相等的话,
那么一定$a[i..j]$满足条件。

如果把所有的$a[1..i]$都储存下来,
当一个数据结构相等的时候,就可以肯定,
我们找到了一个符合条件的值。

综上,时间复杂度可以降到$O(n \text{lg }n)$,
甚至是$O(n)$(如果用哈希表的话)。


\section{cf's 1307B problem}\timetx{2020-02-18 11:42:37.636214}

危\ 我\ 危!
我差点就要掉到绿色等级去了啦!!!
只做了一道题,第二题,第三题完全不会,
说实话还是有点点灰心的吧。

回到这道题来,我认为这道题完全就有误导我的嫌疑!

给定原点$O(0,0)$和目标点$A(x_A,0)$,
又给定了可选的距离$B=\{b_i\}$(可以在二维平面上移动)。
然后要使用这些可选的距离从原点移动到目标点。

要求取最少次数。

开始我的想法是:
如果用$m$表示最大距离,那么如果能用$m$就最好用。
要是$m|x_A$成立,那么很明显$\frac{x_A}{m}$是所求值。
但是不成立呢?我认为可以先以$2m$为单位地开始移动,
到最后很明显剩下的距离有$e\in [1, 2m-1]$,
如果$e\in B$,那么很明显最后一节一步搞完,
不然两部搞完(用三角形两边之和大于第三边,
那么两步可以移动的距离就是一个范围:$(0, 2m]$)。

但是一开始的实例就过不去,因为当$x_A=12$时,
$B={3, 4, 5}$时,我认为的算法给出的答案是:
$5\to 5\to 2$(two step),一共是4步。
而底下的注释告诉我应该是3步:$4\to 4\to 4$。
就是这个迷惑了我,让我以为其他的数字,
也会参与到构建路径中,让我以为有一个最优解,
以至于不得不使用动态规划。

动态规划可是有$O(n^2)$的复杂度,这道题就很明显的堵住了我。

之后看了其他人的答案后:
我开始为他们的代码的短小精悍感到吃惊,
结果看来反而和我一开始的想法有同工之妙。

他们还是用最大距离$m$来解题,
而最后剩下的距离和之前的$m$步长的一步加起来,
肯定在$[m+1, 2m]$,故最后最少必定可以用到两步。
易证明到,在这个分割点之前是最优解,
分割点之后是最优解,而这个分割点本身也是最优分割点
(虽然不止一个最优分割点,但其他分割点的答案
不比它的答案还有优,也就是说,这么分割必得最优解)。


\section{cf's 1307C problem}\timetx{2020-02-18 07:16:08.124380}

老规矩先说题干。
给定一个字符串,找到符合规矩后寻找到的出现次数最多的子字符串。
规矩是原字符串中寻找成等差数列的有限长度下标集合
(其中最短长度是1),然后将其按照下标提取出来。

这道题的陷阱就是这个等差数列了,
这个也正是最迷惑人的东西了,因为题解中最主要的部分和等差一点关系也没有。

以下是题解原文:

We observe that if the hidden string that occurs the most times has length
longer than 2, then there must exist one that occurs just as many times of
length exactly 2. This is true because we can always just take the first 2
letters; there can't be any collisions. Therefore, we only need to check strings
of lengths 1 and 2. Checking strings of length 1 is easy. To check strings of
length 2, we can iterate across S from left to right and update the number of
times we have seen each string of length 1 and 2 using DP.

如上,证明分两步:
一,长度$len_b>2$的字符串$b$出现次数小于等于长度$len_s=2$的字符串$s$:
如果有字符串$b$存在的话,其对应的前缀$s$必定也满足下标成等差数列。
故$b$出现次数必定小于等于$s$出现的次数。
二,长度为一的字符串可能出现次数小于长度为二的字符串的出现次数:
不同下标集合构成的长度为二的字符串,可能前缀都是同一个下标对应的字符,
所以长度为1的字符串可能对应多个长度为2的字符串,
但是如果存在的话,长度为2的字符串却有且只有一个对应的$len_b>2$长度字符串。

如上证明了我们最主要的地方就是分析长度为1和2的字符串。


\section{MIT 6.006 part1(unit1)}\timetx{2020-02-19 14:58:32.463955}

In this course, I am interseting about the finding peak in 1D-gragh or 2D.

So what is the peak in 1D-gragh?
for a number in a 1D-array, if its left $L$(if $L$ exists) is less than it and its
right $R$ also do it(if $R$ exists), then the number is a peak.

there is a $O(\lg n)$ function that can work it out:
firstly, look at n/2 position, if the element is a peak then break the loop;
if it is not, and the left is bigger than itself, we can know than in the
sub-srray $A[0..n/2-1]$ must have a peak element inside. (why?)
else it can look for the sub-array $A[n/2+1..n-1]$.

It is easy to find out the compilex is $O(\lg n)$.

For the question why I assert the sub-array must have a peak at least inside,
the answer is that the sub-array must have a max-element, and if the max-element
is at the edge of the array, then it's easy to know that it is a peak.

In a 2D-graph, the peak is bigger than or equal with its round cell numbers. the
best function to find one of those peaks swiftly is: firstly, look at the middle
column and the find out the max-element of the column, and if the element is a
peak then return the answer, but if it does not and the round cell number $C$ at
column $n/2-1$ is bigger than it, we can say that on the area from column $0$ to
column $n/2-1$ there must be a peak at least.

Why? - if the peak $P$ is on the column $n/2-1$, the peak is must bigger than
$C$ and then is bigger than all numbers on column $n/2$, beacause if it is small
than $C$, the max-element is not it, and we can now look at the max-element of
the area.


\section{cf's 1296D problem}\timetx{2020-02-20 15:30:16.633458}

这是一道简单的D题,我不得不说不亏是我(骄傲)。

有几个关于C++的小知识和之前复习到的小技巧:

第一个是Variable-length array(VLA),还为维基百科做了一点点贡献,
VLA是在运行时分配内存空间给数组的一个语言性质,一般是在C99之后支持。

另外是复习之前的一个小技巧:如果想表达$\lceil\frac{m}{n}\rceil$,
则可以使用:\vb|(m + n - 1)/ n|。

使用标准库函数\vb|std::sort|时,作用于数组\vb|A|时,
作用方法:\vb|sort(A, A+len-1)|。

另外发现宏函数时支持多参数的。


\section{一道数学题}\timetx{2020-02-20 19:37:50.946162}

题目摘抄如下:

设实数$a$,$b$满足$0<a<b$,证明:
$$\frac{\arctan b-\arctan a}{b-a}
>\frac{1}{\sqrt{1+a^2}\cdot\sqrt{1+b^2}}$$

一开始的时候很容易看到可以把原式化为:
$$\frac{\arctan b-\arctan a}{b-a}>
\sqrt{(\arctan a)'\cdot(\arctan b)'}$$

从而会想到使用中值定理,但是该式根本推导不到有用的中值定理上,
而正解是使用反函数的性质。
(说明了看见反函数按理应该使用其性质的一般套路)

不妨令$x=\arctan a$,$y=\arctan b$,则根据反函数的性质有:
(根据原题可知:$y>x>0$)
$a=\tan x$,$b=\tan y$,所以很明显原始等价于:
$$\frac{x-y}{\tan x-\tan y}>
\frac{1}{|\frac{1}{\cos x}\cdot\frac{1}{\cos y}|}$$

进行又一次简单的变换有:
$$(\cos x\cos y)\cdot\frac{x-y}{\sin(x-y)}>|\cos x\cos y|$$

很明显有:
$$\frac{x-y}{\sin(x-y)}>1(y>x>0)$$

故原式成立。

\subsection{中值不成立的原因}
根据中值定理该式$\frac{\arctan a-\arctan b}{a-b}$
很明显有:$\exists \xi$,
$(\arctan\xi)'=\frac{\arctan a-\arctan b}{a-b}$,
所以$((\arctan\xi)')^2>(\arctan a)'(\arctan b)'$。

为了利用中值定理解出该题,则只能通过对$\forall\xi\in(a, b)$,
都有该式成立来证明。但很可惜对于$\xi\in(a, b)$,
$\exists\xi\in(a, b)$可以令其不成立。

比如因为有$0<a<b$,有当$\xi\to b$时,有:
(因为$\arctan x$的导函数在正半轴上单调递减)
$$\lim_{\xi\to b}((\arctan\xi)')^2=
((\arctan b)')^2<
(\arctan a)'(\arctan b)'$$

所以使用中值定理是一种错误的选择,该方法绝对搞不出来正确答案出来。


\section{cf's 1299A problem}\timetx{2020-02-20 23:25:24.150958}

这是一道我不会做的A题。

原题定义了一个函数$f(x, y)=x|y-y$,其中的符号$|$代表了进行位运算。
又定义了一个数组$[a, b, c, d, \ldots]$的值应该为:
$$[a, b, c, d, \ldots] = f(\ldots f(f(f(a, b), c), d), \ldots)$$

现在给定了一个数组,请求得到进行调换顺序之后值最大的数组。

位运算是一个我不太熟悉的东西,但是仔细分析还是很容易发现:
对于$x$和$y$而言的第$i$-th位$x_i$和$y_i$而言,
进行$f$变换之后,新的值$z$的第$i$-th位$z_i$仅仅与$x_i$和$y_i$相关。
因为当$y_i$等于1时,$(x|y)_i$必定为1,所以$z_i$必定为0。
(因为当被减数上第$i$-th位为0而减数上该位为1的话就需要借位,
势必和其他位有关)同理可以得出当且仅当$x_i$并且$\lnot y_i$时,
$z_i$为真,故$z_i=x_i\land\lnot y_i$。

由上因为所有位均满足该运算规律,故有
$z=x\&(\sim y)$。

所以对数组而言,自然也有:
$$[a, b, c, d, \ldots] = a\&(\sim b)\&(\sim c)\&(\sim d)\&\ldots$$

因为$\&$位运算满足交换律,所以原式的值只和第一个数有关。
且其所有被运算的数第$i$-th位为真,其才为真。
那么什么时候需要调换以让该式的结果更大呢?

从最高位开始比较,如果有且只有一个数在该位为一,
那么让它成为第0位。因为其他位上都会取反变成1,
而这个数本身的位上的数也保留下来了,从而取到最大。
(太棒了)

可是我对位运算不太熟悉,下面是别人的位运算语句:
\begin{lstlisting}[language=C++]
#include<bits/stdc++.h>
using namespace std;
 
int main(){
    int n;
    cin >> n;
    vector<int> a(n);
    for(auto& x: a)
        cin >> x;
    for(int bit=30; bit--;){
        if(count_if(begin(a), end(a),
                    [&](auto x){return x >> bit & 1;})==1){
            swap(a[0],*find_if(begin(a), end(a),
                               [&](auto x){return x >> bit & 1;}));
            break;
        }
    }
    for(auto x: a)
        std::cout<<x<<' ';
    std::cout<<'\n';
}
\end{lstlisting}

首先有巧妙的位运算小窍门:\vb|num >> bit & 1|,
它可以确认\vb|num|的第\vb|bit|位是否为一。

其次有有意思的\vb|count_if|和\vb|find_if|。

再然后是lameda函数。之前我写过的,这里就暂且不表了。


\section{cf's 1321B problem}\timetx{2020-03-02 13:43:23.543056}

在昨天的cf竞赛中,我又挂在了B题,原题是这样的:
有人想要去旅行,每个城市都有相应的数值,
并且第$i$个城市的数值为$d_i$,当且仅当第$j$座城市的数值满足
$b_j-b_i=j-i$时,她才可以从城市$i$走到城市$b$中。

我一开始这么想的:点,路,有价值,这不就是有权图的最长路径问题吗?
一种有向无环图的最长路程是具有最优子结构的,无论自底向上还是至顶向下,
都是可以满足动态规划的。

我于是刷刷的写完了,虽然途中遇上了几个\vb|core dumped|说是下标溢出了,
但是还是找到错误并且搞好了。

但是罕见的遇上了\vb|memory limit exceeded|,
其实我现在还是感觉很奇怪的,因为怎么想就只是需要用$O(n^2)$的复杂而已,
还主要是因为储存图才这样子。

但是昨天于立恒跟我说了他的思路的时候,我感到柳暗花明又一村,
发现当且仅当$b_j-j=b_i-i$的时候才会有两个城市可以连接在一起。

所以说最好的方法就是建立一个$map$,当$b_n-n=C$时,让$map_C$
增加$b_n$。而$map$中最大的值也就是这道题的答案。


\section{python的多项式拟合}\timetx{2020-03-03 20:57:50.277423}

之前的话我想要获得一个动态的页面,可以根据不同的窗口大小来动态匹配文本大小,
比如说在一些窗口宽度大的地方,一般的像素密度比会小一些,为了显示合适的字,
需要把字调小(比如说,\vb|16px|),有时候在小屏幕上面,则需要把字号调大,
比如说调成\vb|32px|。

我不禁有些疑惑,像素到底是什么呢?比如说虽然移动端字的\vb|px|值调大了,
但是实际上的字的大小却变小了,虽然说是变小了,但是设备像素,本身就高,实际的物理
尺寸却没有随着密度变得过于小(实际上移动端的密度对于电脑来说大得吓人)。

实际上,在真实的设备像素和众多的html长度之间,隔着一层CSS像素层,是一种相对像素。
因为在移动段的ppi过于小,而且因为对应的页面指定的\vb|px|不变。
所以说在移动端,我们需要使用更多的像素来表示一个对应的CSS像素。

其中设备像素叫做DP,而CSS像素\ldots\ldots 还真叫css像素。

描述物理像素和css像素之间的关系的式子就是DPR,设备像素比。

而还有PPI来描述每单位的像素密度,说实话PPI的作用不大,因为不同的设备上并不是说,
各各单位一致就可以获得一个完美的关系,一般手机屏幕小,PPI高,
元素的绝对宽高应该是相应会减少。

所以说我们可以根据宽度的css像素来确定确定根的字长。

按理来说,在经典视角下我们看到的网页都会差不多一样大,
因为css像素是在经典使用视角的长度来进行确定的。

但是一般来说我们会适当的想减少信息大小来使得信息更加清楚。
所以针对不同的屏幕css像素宽度,代表了最少可以显示的信息流,
信息流相对较小,那么我们更应该减少信息流来获得很好的观看体验。
来达到大小差不多的样子(应该?)

为了只使用一个表达式来表达不同情况下的多项式拟合,
使用python的numpy无疑让人心情愉悦。通过使用\vb|numpy.polyfit|和\vb|numpy.poly1d|,
很轻松就能得到一个适当的曲线。


\section{在wsl中编写含CJK字符的文章}\timetx{2020-03-04 18:07:35.291983}

我用的wsl是Dedian发行版,尝试使用pandoc和xelatex来转换md文档时发生了不少的事情,
比如说不支持中文,这是因为没有设置\vb|\CJKmainfont|,而且必须要设置正确,
不然到时候调用的时候会报错说没有这个字符文件。

字符文件的名字和它的文件名不一致,所以说可以会含有空格,使用\vb|fc-list|命令,
可以获得它的正式名称。在这里可以安装一些字符包(一般来说都是开源的),当然,
字符包的名称,字符文件的名称和字符文件的文件名是三个不同的概念。

另外有一件事情有点邪门,明明来说wsl是可以让exe应用程序访问的,但是pandoc好像不行。
这就奇了怪了。


\section{\LaTeX 的旋转表格}\timetx{2020-03-05 13:51:19.334722}

因为表格太长了,所以说有点时候需要把相关的表格横着放。

google了很容易发现,有一个叫做\vb|rotating|的package提供了
\vb|sidewaystable|环境,提供了类似于自带的\vb|table|功能,
不过不同的只是它的表格是颠倒过来的。

我这一次使用了\vb|[p]|选项让它放在新的一页。还使用了
\vb|\caption|来生成图片,使用\vb|\label|来为引用服务。


\section{令cmd接受参数}\timetx{2020-03-05 18:29:41.111661}

在wsl中可以通过cmd.exe来进入命令行,但是也可以传入参数来运行\~{}

使用类似于\vb|cmd.exe /C "command line"|来驱动程序。
传入的命令完全是在cmd.exe中运行,所以说也可以读取到cmd.exe中的PATH。
通过这个完全可以用来让本来运行在windows的程序套个sheel壳子从而在wsl中运行\~{}

开心中\~{}


\section{在wsl中启动jupyter-notebook}\timetx{2020-03-06 18:11:44.615188}

为了在wsl中使用jupyter,那么一定需要一个浏览器。
而wsl一开始就只有终端,并没有提供图形界面(我很喜欢)。为了使用它,
可以使用\vb|jupyter-notebook --no-browser|,它提供了一个端口和
一长串token。使用这两个就可以通过网络协议来使用!太棒了!

但是有的时候可能之前有东西占住了端口,\vb|8888|是不行的了,
也可以使用一个选项\vb|--port=other-port|来自己定义一个端口。


\section{latex关于\texttt{\textbackslash end\{...\}}的报错问题}
\timetx{2020-03-07 12:14:05.564314}

做数学作业的时候latex说有一个错误发生在\vb|\end{...}|上,
我百思不得其解,后来发现是有一个环境只写了\vb|\begin{...}|,
但是没有写\vb|\end{...}|,这就是原因欸。

所以说报错的地方可能和真实错误的地方有一定距离。


\section{cf's 1321C problem}
\timetx{2020-03-07 23:55:00.900381}

我好像又把一道题目想复杂了欸。

题意如下:给定一个字符串,如果相邻的字符恰好比它本身小,
那么将它移去。

答案是:从\vb|'z'|开始遍历到\vb|'a'|,如果遇到字符串中是
遍历的这个数,而且恰好满足条件,那么将它移去。(欸!
是简单的模拟!)

当然也不是什么都没有学到。我学到了\vb|string|的一个
method,\vb|erase|可以用来删除字符。

当然每一次找到了之后就从头找起\ldots\ldots(妈呀我感觉好气!
我真的是一个大笨蛋!)


\section{cf's 1316B problem}
\timetx{2020-03-08 14:21:58.394338}

老规矩先说题意。
这道题给定了一个字符串(长度为$l$),和一个指定的动作:\vb|reverse|子字符串。
对数$k$而言,需要将它从$[0, k)$一格一格的走到$[l-k,l)$,
每一次都将范围内的子字符串进行操作。

说说学到的几个操作:

定义别名!如何为一个STL定义别名呢?使用using:
\begin{lstlisting}[language=C++]
template<typename T>
using vec=vector<T>;
\end{lstlisting}

需要注意的是\vb|vec|还是需要使用\vb|<type>|,我好像有点理解了\vb|template|的工作原理。

其次有一点需要注意,\vb|string|是不能接受字符的,只能接受字符串。

话说,这道题对复杂度的要求还真是宽大啊。


\section{shell脚本中的单引号和双引号}\timetx{2020-03-09 14:47:36.004608}

我想要自己搭建一个独特的promat,所以为了自己的路径提示,
我定义了一个函数,并且使用了双引号来定义我的\vb|PROMAT|
变量。

但是这样子有一个地方出乎我的意料,当我cd到另一个目录的时候,
promat应该进行更改,但是事实上却没有按照我的意图。

在shell脚本中双引号是会首先把本身的变量和运行命令一一运行和替换,
而单引号不会。但是zsh本身在接收一个字符串之后会每次都又运行一遍,
所以说我们应该传入一个没有被替换的文本串\~{}


\section{cf's 1312C problem}\timetx{2020-03-10 16:02:05.549992}

给定一个目标数组,
然后需要用指定的$base$通过$base^{p_0}+base^{p_1}+\cdots$来表示,
如果数组里面每一个数组中的每个数构成的$p_0$,$p_1$,$\cdots$互不相同,
而且任意两个数中的$p_i$互不相同就可以说这个数组是一个好数组。

通过这个题我知道要注意两件事情:

\subsection{注意上限}
原题中说的是数组中的数的取值范围是$[0,10^{16}]$,
但是我还是傻乎乎的使用了\vb|int|(不然这道题我就过了啦)。

\subsection{关于本应该输出整数的浮点算法}
在使用对数的时候,按理来说有点时候应该输出一个整数的,但是却输出了浮点数,
而且之后无论本来应该是整数还是说是小数都需要过一遍\vb|floor|函数,
所以有些整数比如说\vb|4.9999998|就会很不好意思的变成一个不应该变成的数。
下面有三种方法:

\begin{enumerate}
\item 加上配重,比如说\vb|1e-4|或者\vb|1e-5|之类的。
      但是这种办法总感觉蛮容易被别人hack的。
\item 检验,得到对应的整数的时候进行检验,看看得到的符不符合预期。
\item 使用整数算法,比如使用除法得到对应的对数的解法。
\end{enumerate}


\section{链表的使用}\timetx{2020-03-11 08:44:46.127475}

刚开始想要建立一个链表,但是其实上对C语言的管理不清不楚,以为和python一样,
只要有引用计数,就会一直存在,但是不然,C语言的话,只要是变量离开了它的定义域的话,
就会被销毁,不仅仅是回收栈上面的内存,连对象的折构方法也会被一并调用。

我一开始用一个指针指向了一个局部变量,结果一出局部定义域,
这个变量的空间就被标记为自由的内存栈空间,但是值却没有初始化,只是被标记了而已。

之后的话我又重新声明了一个局部变量(在另外一个定义域之内),好家伙,结果它用了前面一个的地址,
链表就形成环了啦!我百思不得其解。

\subsection{动态的分法}
在C++中一般是通过\vb|new|和\vb|delete|来管理内存的新建和释放。
在C语言中则是用\vb|malloc|和\vb|free|。

语法格式差不多是:
\begin{lstlisting}[language=C++]
T* var = new T <initialization>;
delete var;
\end{lstlisting}

以上哈,还有对于空指针\vb|nullptr|的话,\vb|delete|是安全的操作。


\section{cf's 1324D problem}\timetx{2020-03-13 16:58:11.221877}

终于有一次让我做到了D题(虽然还是没有做出来),
这道题是给定两组数据,分别是$\{a_1,\ldots,a_i,\ldots,a_n\}$和
$\{b_1,\ldots,b_i,\ldots,b_n\}$,如果下标$i$和$j$满足:
$$a_i+a_j>b_i+b_j\Rightarrow (a_i-b_i)+(a_j-b_j)>0$$
求出可能的下标组合的个数。

我用笨办法来做,先令$c_i=a_i-b_i$,那么对于每一个$i$而言,
我从正数找到最后一个可以让$j$的值满足$c_i+c_j>0$。那么累加起来,
可能的组合数就求出来了(大体如此)。

答案是怎么做的呢?好吧,既然你要找到一个令$c_j\ge 1-c_i$的$j$值,
那么说二分搜索它不香吗?(指\vb|lower\_bound|)
好吧一下子把$O(n^2)$变成了$(n\text{log}n)$啦。


\section{关于map<bitset>的使用}\timetx{2020-03-14 14:45:30.148432}

众所周知,\vb|map|作为一个模板一般接受四个值,而我一般只给定前两个,
而后面的值让模板自动补全即可。

今天我打算用\vb|bitset|作为键,但是编译器好像看不起\vb|bitset|,
潇潇洒洒给我了一千字的长文,比我自己写的代码还多,从头看到尾,
字里行间都是\vb|operator<|,我寻思咋地啦,我没有用到小于号啊,
你别报其他头文件的错啊,你报这个的,但别人不听,还跟我抬起干来了。

在stack overflow上面使用了一些小花招,搞到了答案,一看,
原来作为红黑树是要通过小于号来进行比较的,才能产生对应的树。(嗨!
我也不是不知道,但还真没想到那方面去)所以说,我转念一想,
这个的意思是说,\vb|bitset|不支持大小比较咯?(嗨,这不废话吗!
这玩意?宁还真把它当数啦)

为了比较它这个类型,我们必须给map多赋值一个类型,这个类型的对象,
必须可以调用来比较两个key之间的大小关系。

差不多吧,我边写边抄,差不多代码如下:
\begin{lstlisting}[language=C++]
template<size_t N>
struct bitset_less{
    bool operator() (const bitset<N> & a, const bitset<N> & b){
        if(N == 0) return true;
        size_t i = N - 1;
        do {
            if(a[i] ^ b[i])
                if(a[i]) return false;
                else return ture;
            --i;
        } while ( i > 0 )
}
\end{lstlisting}

将\vb|bitset\_less|传入代码即可。

在抄代码中首先使用$i$从$N-1$一直循环到$0$,但是我忘记了\vb|size\_t|
作为一个无符号正整数,是没有负数的,所以我设定的大于零这个条件是很成立的。
(我在stack overflow上面问的第一道题)

