\section{cf's 4A problem}
\timetx{2020-01-12 09:19:39.244631}

divide a even ingeter into two parts, each of them is even

这说明了这个整数能被2整除(因为$\text{even}+\text{even}=\text{even}$),且不能为2.


\section{Python的编码问题}\timetx{2020-01-12 12:17:27.624951}

尝试用\vb|print('...', file=out\_put\_file)|来进行文本输出,
发现输出的文本不是用\vb|UTF-8|来编码的,
而是好像用的国标(国家的标准?)

在\vb|str|中字符是用\vb|unicode|来编码的,是没有被\vb|encode|的二进制数据,
输出到文件,输出的内容编码是由系统的配置决定的。

猜想:\vb|open|时会指定\vb|encoding|可以解决这个问题。

\vb|open|时指定\vb|encoding|会有带有编码信息的属性的文本对象。
因为有状态的影响,所以该对象的\vb|write method|会用编码下的二进制实现去写文件。

而\vb|print|可以指定\vb|file|,但是它(应该)不会读取文件内部的编码状态,
(这个时候我想起来了,有时用\vb|u8|编码的时候好像会在文件头留下一个特别的标记?)
而是直接用系统指定的编码格式去输出二进制数据去写数据。

猜想:\vb|print|是把文件输出到\vb|sys.stdout|上(应该,至少意思是这样)。
输出的环境是由系统的配置/环境决定的,为了环境能够正常显示文本数据,
就应该会用配置下的文件编码格式来编码\vb|Unicode|的数据流,
这也是为什么它不会去读取文本文件的编码状态的原因吧?
(甚至还没有参数可以调节输出的编码状态,这样)


\section{matplotlib的字体输出}\timetx{2020-01-12 12:19:30.764825}

这个问题我之前也记在白皮书上了的,但是在写一点也ok。

\vb|matplotlib|以什么字体输出是根据\vb|cont.family|,\vb|font.serif|,\vb|font.sans-serif|等决定的。
(\vb|text.usetex = True|是指排版用\TeX ,字体用配置指定的,还是全部由\TeX 自己决定的?)

\vb|font.family|指定输出的字体的字系,有诸如\vb|serif|可以选择。

一般\vb|serif|,中文字体感觉不错的有经典的\vb|SimSun|(宋体),
而\vb|sans-serif|,用于编程的\vb|Consolas|也不错。

关于修改配置,可以在文件中修改,或者用\vb|plt.rcParams['xxxx.yyyy'] = data|来修改的。
(当然可能有其他的修改输出格式的办法)

在\vb|axes.texta|中,用\vb|family|参数可以临时指定字系。应该也可能有其他的方法可以。

