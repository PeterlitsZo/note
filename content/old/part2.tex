
\section{matplotlib的cmap}\timetx{2020-01-12 12:20:10.551049}

之前网上的示例用到了`\vb|RdYlGn|'的\vb|cmap|,通过\vb|plt.get\_cmap|来获取的。
之前以为\vb|RdYlGn|的中间是白的,以为它好像是:
$Red\to Orange\to White\to Light green\to Green$的样子。
结果看了才知道并不是这样子的,它中间反而是黄的,是\vb|Yl|的样子,有一说一,
确实,这也太白了一点。

也有颜色表,比如\vb|tab20c|,一格一格的。

不是很确定,返回的\vb|cmap|对象可能调用一个元素为\vb|int|或者\vb|float|的可迭代对象,
然后返回对应的颜色列表。


\section{使用TikZ创建条件图}\timetx{2020-01-12 12:20:53.074716}

我最近非常想要设计一个语法统一优美的作图语言,我简直被\vb|python|宠坏了。

唉......

先提一句:\vb|\tikz\fill[orange](1ex, 1ex) circle (1ex);|
可以画个圆。(这个\vb|TikZ|太麻烦了。

而\vb|\tikz \draw[->] begin -- end;|可以用来画箭头。

今天在知乎上面逛了一圈,有人说可以用\vb|python|来搞个UML图,试试就试试。

p.s. 我现在用的是\vb|graphviz|的\vb|dot|语言。
还可以吧,但是总是有点麻烦。语法格式也很不统一。


\section{\LaTeX 中的对齐问题}\timetx{2020-01-12 12:21:33.898432}

\LaTeX 中有两种对齐方式:(1) 环境对齐,(2) 命令对齐。
其中环境的话是上下都空了个间隔,用命令会好一点。
但怎么说呢,命令的话会作用到一个\vb|\par|。

注:VsCode中用\verb|$\$$nunber|来表示捕获的分组。



\section{Python中的setup.py}\timetx{2020-01-12 12:22:21.429750}

通过setup.py文件,可以使用命令\vb|python setup.py install|来安装
包。(注,要在setup.py目录下运行才会ok,也就是说,\vb|cwd|是setup.py的
父目录)

也可以使用\vb|setuptools.setup(...)|的那啥来指定外部名,之后就可以在外面直接用了。


\section{python的pyyaml问题}\timetx{2020-01-12 12:22:55.901702}

直接用pyyaml会有点问题。现在我是在用ruamel.yaml的第三方包。

效果不错,但是还有要改进的地方。我想去参与开发这个项目。有一说一,这个项目还不错的说。



