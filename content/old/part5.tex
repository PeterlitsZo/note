
\section{关于undefined reference to `WinMain'}\timetx{2020-01-24 05:16:31.300438}

因为如题目这样的\vb|Compilation Error|,导致了我重新安装了一边g++,
因为一直在纳闷明明没有使用WinAppiliction的我,是怎样搞出的``WinMain'',
结果看了一下(在半个小时之后了),是我的main函数写错了,写成了maim。


\section{cf's 1294B problem}\timetx{2020-01-25 06:07:54.481951}

这是一个简化的机器人走路的题,只能向上或者向右走。
并询问你是否能够完成任务和完成任务的方法。

分析后可知,当机器人处于一个点时,它只能到达以它为原点的第一象限区域。
也就是说,每一个点都必须要在前一个点的右上方。
易知,对于一个点来说,除了其右上方的点外,它相对与其他点来说都是在右上方向的。

所以对其他的点来说,令它们以$(x_i, y_i)$排好序后,
就分为两种情况。第一种,在同一个$x$上,排在后面的都是在它的上方,
第二种,在不同的$x$上,其点的$y$坐标很明显会大于这个$x$坐标上最上面的一个点的$y$坐标。
(如果不是,就无法完成任务)

\subsection{little bug}
其一。

对于\vb|*a.b|,我不是很清楚它们之间的优先级\ldots\ldots
它会先调用点运算符,之后在调用\vb|*|号运算符,所以要使用\vb|(*a).b|的形式。

其二。

\vb|int|和\vb|char*|是不能相乘的(我知道这是屁话,但还是希望它能搞个\vb|string|出来。
然后\vb|int|和\vb|string|也是不能相乘的,这个倒是超出了我的预算。
要使用重复的字符串,应该使用:\vb|string(int n, char c)|来生成。

其三。

通过上面的那一回事,我发现了盲点!
\vb|type t(a, b)|和\vb|type t = type(a, b)|应该是等价的才对。

其四。

关于\vb|for auto|,它有这种形式:\vb|for(auto& a: a_s)|,
其中的\vb|a|,其实是\vb|(*iterator)|,而不是迭代器。

\subsection{关于其他人的答案}
其他的都差不多,但是好像\vb|pair|是可以比较大小的,
所以标准排序算法\vb|sort(type_t a.begin(), type_t a.end())|是可以支持的。
这就让给\vb|pair|排序成为了可能。


\section{cf's 1294C problem}\timetx{2020-01-25 13:49:00.375339}

这个题的题面是为了让一个数$N$分解成三个不相等的数$a$,$b$,$c$。
即有$N=a\times b\times c$,而且$a$,$b$,$c$三个数互不相等,且所有数大于等于2。

不妨设$a < b < c$,所以$N=a\times b\times c>a^3$,
即有$a\in[2,\lfloor\sqrt[3]{N}\rfloor]$。

\begin{lstlisting}[language=C++]
int result[2], index = 0;
//(index = 0) -> (i^3 < N); (index = 1) -> (i^2 < N);
for(int i = 2; pow(i, 3 - index) < N && index < 2; ++i)
    if(N % i == 0)
        result[index++] = i,
        N /= i;
if(index != 2)
    cout << "NO" << endl;
else
    cout << "YES" << endl << result[0] << " "
         << result[1] << " " << N << endl;
\end{lstlisting}

当在\vb|for|循环中,可以保证如果\vb|result[0]|和\vb|result[1]|能正确产出,
\vb|num|也能成为所谓的\vb|result[3]|。

所以说判别标准就是所谓的\vb|index|是否为2,否则就说明了至少有一个\vb|result|
是没有被正确产出的。


\section{cf's 1293A problem}\timetx{2020-01-28 09:05:51.832202}

本题给出了连续离散集合$A=\{1, \ldots, n\}$,
同时也给出了非连续的离散集合$B=\{b_1, \ldots, b_k\}$,
且有$b_i\in A, i\in [1, k]$。

开始给定了初始数字$s\in A$,然后求出$i$,对于$I=s+i\text{ or }s-i$,
如果对于$i$有$\exists I, I\in A\text{ and } I\notin B$,则求出对应的最小的$i$。

很明显,当$s\notin B$时,$i=0$。

我是通过\vb|set|来作为$B$的容器(以减少查询的时间),
然后从$i = 0$开始查询,看看对于$I\in A$是否满足$I\notin B$,
当有满足的则返回$i$,此时的$i$即为最小的$i$。

另外我发现了一点,在C++中,只有0才代表了\vb|false|,
其他的数都代表了\vb|true|,即使是负数也不例外。

\subsection{其他代码}
对于\vb|set|,我猜测\vb|find|和\vb|count|的复杂度应该差不多。
所以在不支持C++20的编译器来说,
\vb|count|和\vb|contains|在某些方面表现了惊人的可代替性。


\section{cf's 1292 problem}\timetx{2020-01-29 09:02:59.956693}

这道题构建了一个$2\times n$的地图。
在$q$次输入中,$i$-th的输入为$(r_i, c_i)$,
则在地图上添加或移除相应位置的障碍,且求出是否有路径可以从$(1, 1)$格一直走到
$(2, n)$格的位置。

我马上就想到了图的广度搜索,但很明显不对。

然后发现模拟行动路线,走到$(r, c)$位置上时,
如果有障碍物在$(r+1, c)$或者$(r+1, c+1)$的时候就明显无法走通。
对于每一次输入,都要完整的走一遍,
最坏的时间复杂度是$O(n)$。果然\vb|time limit|了。

之后对每个障碍物都看看是否周围有不该存在的障碍物。
时间复杂度就变成了$O(i)$了,随着输入次数的增加而增加。
还是没通过。

\subsection{他人答案}
在看到了其他答案之后,我发现可以使用一个值\vb|barr|来表示通畅度。

在我做够题之后再想想这道题吧。


\section{cf's 1294D problem}\timetx{2020-01-30 04:29:07.199743}

这道题有着严格的时间限制。我小心翼翼的把它优化成了$O(s)$的时间复杂度,
但是依然卡在时间上了。而且最优情况下的时间复杂度是$O(1)$的说\ldots\ldots

\subsection{他人答案}
我看到有人用Python做出来了,我才发现我就是个笨比。

我分析一下我的想法,就是每一次循环的时候,都从头开始算,
虽然有辅助记忆的变量,但还是解出不能。

而看看其他同学的答案,发现他们是在每一次之后接着上一次循环继续做。
这样所用的时间就会少一些。

另外我发现使用\vb|printf|和\vb|scanf|的代码会比不使用的代码,
运行时间会少十倍。(我真的震惊到爆欸)

这么说的话,虽然不喜欢混用C和C++的风格,
但是为了能通过,还是使用C风格的会运行的快一点。
(希望之后不会逼我使用大整数而不是模板)


\section{puts, printf and fwrite}\timetx{2020-01-31 02:28:52.041320}

(感觉我已经走火入魔了)

首先输出一个字符的时候,其实最快的是\vb|putchar|,
但是要输出字符串的时候,不输出格式化字符串的时候,
函数的速度有:\vb|fwrite|$>$\vb|puts|/\vb|fputs|$>$\vb|printf|。

其中\vb|fwrite|的函数声明是:
\begin{lstlisting}[language=C++]
size_t fwrite(const void *ptr, size_t size, size_t count, FILE *stream);
\end{lstlisting}

而根据stack overflow
的一篇回答\footnote{https://stackoverflow.com/questions/2454474/what-is-the-difference-between-printf-and-puts-in-c/2457714\#2457714},
可以看出它省略了调用\vb|strlen|的函数开销。
而相较与\vb|printf|,它更省略了其他的开销。


\section{cf's 1295B problem}\timetx{2020-02-01 02:14:27.397813}

又是一道没有做出来的题目\ldots\ldots

\subsection{最小同余正整数}
为了得到$n$关于$m$的余数,自然是下意识使用$n (\text{mod} m)$,
但是在C++中使用\vb|n \% m|,
却可能不是想要的正整数结果,比如:
\begin{lstlisting}[language=C++]
cout << -1 % 100 << endl;
// -1
\end{lstlisting}

原因是为了将求余推广到整数上,有以下的等价关系:
\begin{lstlisting}[language=C++]
//n > 0, m > 0;
n % m;
n % (-m) == n % m;
(-n) % m == -(n % m); //abs(n % m) < abs(m);
\end{lstlisting}
很明显当$n<0$时,求出来的就不是正整数了,
而为了求出最小同余正整数,根据求余的性质,
只需要在求余之后加上$|m|$就可以了。
唉,如果它和Python学学就好了,
在Python中\vb|-1 \% 23|就是22。
而在C++中不得不用:
\begin{lstlisting}[language=C++]
int atom(const int& n, const int& m){
    int a = n % m;
    if(n >= 0) return a;
    else return a + abs(m);
}
\end{lstlisting}
来作为结尾。

\subsection{寻找元素}
在\vb|std::set|中寻找元素,可以简单的使用如\vb|count|,
或者\vb|find(set.begin(), set.end(), item) != set.end()|,
如果在C++20中,还有method \vb|contains|来寻找。

而在\vb|std::vector|,也只能使用
\vb|find(vec.begin(), vec.end(), item) != vec.end()|
来寻找元素。

当然也可以定义一个模板函数来简化它。

\subsection{整数同号}
为了得到两个数是否同号的布尔值,
可以使用\vb|std::signbit|,
对于整数有:
\begin{lstlisting}[language=C++]
signbit(0); // false;
signbit(1); // false;
signbit(-1); // true;
\end{lstlisting}



\section{cf's 1296A problem}\timetx{2020-02-08 11:07:20.858709}

为了给数组赋值,可以使用:
\begin{lstlisting}[language=C++]
int len_arr;
cin >> len_arr;
vector<int> arr(len_arr);
while(len_arr--)
    cin >> arr[len_arr];
\end{lstlisting}
或者使用\vb|for auto|结构进行遍历:
\begin{lstlisting}[language=C++]
for(auto &i: arr)
    cin >> i;
\end{lstlisting}

我在做这个题的时候,一开始使用了\vb|arr[len_arr-1]|代替了\vb|arr[len_arr]|,
之后就理所当然崩溃了,原因是减一的操作是在判断之后,
所以说,在循环体中\vb|len_arr|的值最小是0。

另外为了求取\vb|sum|值,
应该要使用\vb|for auto|语法或者说是使用\vb|accumulate|,
不过\vb|fot auto|使用的时候会需要一个值,
而另一个则可以直接返回值。


\section{cf's 1303A problem}\timetx{2020-02-14 16:09:05.315418}

给定一个由0和1构成的数值,为了让数的1聚集在一起,
求最少要删除的0的个数。

我的思路是,把首末两端的0给去掉,然后\vb|count|1的个数,
即得到答案。

\subsection{我的答案}
我设置了一个标志,用来判断是否在首末为1的子串中,
如果标志为真,如果遇到0则把\vb|temp_cnt|自增,
不然就令\vb|cnt += temp_cnt|,
这样也能避免末尾的0被归入计算中。

\subsection{其他人的答案 - 1}
\vb|string|有一个方法是\vb|find|,
它会接受一个字符或者字符串,和初始位置(默认为0),
返回被找到字符的位置。

利用它,可以在寻找两个1之间的0的个数。
比如给出初始位置为$n$,而返回值为$m$,
则如果$n = m$,则说明两个1之间没有空隙,
有$m - n$个零,并让$n += 1$。

而如果$n < m$也同理。

\subsection{其他人的答案 - 2}
\vb|string|除了\vb|find|,也还有\vb|rfind|,
所以把首末的1的位置找到并递归就明显好做得多。

