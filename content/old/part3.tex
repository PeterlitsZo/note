
\section{cf's 71A problem}\timetx{2020-01-12 09:50:00.711916}

在C++中因为对象没有特殊方法(当然,构造方法和解构方法除外,还有运算符重载,如果这也算的话)

所以说并不是所有对象都可以转化为字符串的。从另一个方面来讲,如果过于底层的话,的确不需要用到这种方法
(比如Python的\vb|\_\_str\_\_|方法)。

(话说,我是在说服自己吗??)

对于基本的数据,\vb|std::string|提供了特殊的方法,
在std的命名空间里,提供了方法\vb|std::to\_string(number\_type)|,

另外,如果输出要换行的话,也不能忘掉。

p.s. 在上一题中,我也要时刻注意输入数据的范围。


\section{cf's 118A problem}\timetx{2020-01-12 15:35:41.865311}

 遇到了几个有意思的问题。

 首先时如何把字符串转换为小写的字符串。
 在python中只需要使用\vb|str.lower()|就可以得到一个拷贝了。
 (谢天谢地,我现在特别怀念python)
 但是很明显这在C++中时行不通的。

 第一种转换方法是使用\vb|transfrom|函数再配上一个lambda函数。
 transfrom是由标准库algorithm提供的,如关于它的介绍网站%
 \footnote{在https://en.cppreference.com/w/cpp/algorithm/transform中}%
 说的那样,它的几个声明之一是:

 \begin{lstlisting}
template< class InputIt, class OutputIt, class UnaryOperation >
OutputIt transform( InputIt first1, InputIt last1, OutputIt d_first,UnaryOperation unary_op );
\end{lstlisting}

它会把unary\_op作用到这些$[frist1, last1)$上去,而输出到$[frist2, +\infty)$上去。

而lambda函数,最简单的形式是这两种:

\begin{lstlisting}
[ captures ] ( params ) { body }
[ captures ] { body }
\end{lstlisting}

回到主题来,这个函数就是为了封装住来自cctype的函数\vb|std::tolower|的函数原型是
\vb|int tolower(int ch);|

以上是第一种方法。

第二种方法是使用boost库,不表。

之后遇到了有多个使用或和等于的逻辑判断符,换个思路,其实用set可能也是一个不错的思路。
有一点很有意思,如果找不到,一般会返回该容器的\vb|.end()|的值。



\section{cf's 85A problem}\timetx{2020-01-13 01:32:53.936011}

和上次那道关于字符串的题很像。都涉及到了把字符串转换为相应的小写形式。

关于转换的函数,应该是下列的样子:

\begin{lstlisting}
std::string lower(std::string str){
    std::transform(str.begin(), str.end(), str.begin(),
            [](unsigned char c){return std::tolower(c)};
    return str;
}
\end{lstlisting}

我自己在使用中的时候,lambda函数没有加上\vb|return|语句,下次一定。
(lambda函数是一个黑盒,必须要有输入有输出)。


\section{关于C++中的using和typedef}\timetx{2020-01-13 17:25:19.221487}

总是因为泛型的原因要声明很长的变量类型,但是其实有为变量类型设置别名的方法。

第一点是使用\vb|typedef|来重命名变量。格式如下:

\begin{lstlisting}[language=C++]
    typedef org new;
\end{lstlisting}

另外的方法是使用\vb|using|来进行命名变量的工作。
这是C++11起才开始支持的,语法会更好一些,更统一化一些。
另外它还支持模板操作。
普通的用法是这样:

\begin{lstlisting}
    using new = org;
\end{lstlisting}

或者说这样:

\begin{lstlisting}
template <typename T> using my_type = whatever<T>;

my_type<int> my_var;
\end{lstlisting}

这样相较而言,typedef就好像是宏定义的一样(当然并不是)。


\section{Python中的\_\_new\_\_方法}\timetx{2020-01-14 09:28:06.576783}

在Python中的\vb|__init__|方法一般只是用来设置属性用的。
换言之,\vb|__init__|只是在使用\vb|__new__|后获得对象后给对象加属性
而使用的特殊方法。

所以说,真正可以获得对象的方法,还是要用到\vb|__new__|的特殊方法。

而一般的类设计是不需要定义\vb|__new__|特殊方法的,
原因是对于它们来说继承属于\vb|object|类的\vb|__new__|方法就OK了
(事实上是没有object这个类型的声明的,
但是关于这个概念,每个设计类的人都要理解,因为它是用Python的解释器实现的,
是一切类的基石,就像内建类型一样,不过话说,object也的确是内建类型)。

所以可以认为\vb|__new__|是特殊方法中的特殊方法。
是调用类之后的之后第一个被调用的类方法。
而它生成的对象更是其他方法的基础。

因为这个,\vb|__new__|不同与其他方法一样,反之,
它被传入的第一个参数是\vb|cls|,是类对象,
而不像其他方法一样传入的是实例对象,也就是\vb|self|。
在最后的最后,\vb|__new__|会返回一个类对象所对应的实例对象。

从类到对象,一般而言只需要调用\vb|object.__new__(cls)|方法就可以了。
如果想对自己的对象加入更多的细节,
都可以在自己的类下的\vb|myclass.__new__|定义余下的,
甚至还可以实现元类。

当然如果有选择的话,在自己的对象下实现\vb|__init__|来定义,
这永远是最优选择,就如Python之禅所说的一样。
当然从另一个方面来看的话,我们会发现\vb|__init__|并没有我们想象的那么必不可缺。
很多时候甚至可以找到其他的办法来实现\vb|__init__|方法所能办到的。
但是其他的方法一般来说完全没必要,
简单的\vb|__init__|已经简单得够招人喜欢了。

在\vb|__new__|中,为了使用父类所已经完成了的工作,
也可以使用强有力的\vb|super()|,它的参数还有几个可以传值呢,
从而为继承提供了更好的基础。


\section{Python中的hash}\timetx{2020-01-15 02:24:11.383948}

想要完成的功能:
为不同的类型的对象进行一个独一无二的hash标记。

为了完成这个功能,我看了看Python的标准库,\vb|hashlib|库。
\vb|hashlib|库为hash算法提供了良好的支持,
比如说要计算my-object的sha-1的哈希值,可以这样:

\begin{lstlisting}[language=Python]
m = hashlib.sha1()          # 新建sha1对象
m.update(str(my_object))
m.digest()                  # sha-1值
m.hexdigest()               # 表示为16进制字符串的sha-1值
\end{lstlisting}

(?):update可以直接接受字符串吗?它不用被编码吗?


\section{cf's 1288A problem}\timetx{2020-01-15 07:11:57.073754}

看了看其他人的回答,发现了有人求最小值的时候没有用循环,
还用到了莫名奇妙的变量,我想,莫非这是用导数算出来的吗?

于是我找到了导数等于0的点,然后带入原函数,发现了眼熟的变量......
我的天呐,我竟然看到别人用数学解题,这算不算优化过度?
我蛮喜欢数学的,但是我也喜欢简单直接的可读性,
我就佩服佩服一下了。
我还是用循环来做题吧。

还有人写了一行Perl代码,而相比之下我写了31行才搞出来。
先生真若神人也。

在我自己的代码里也有不少的问题,希望在C++里变量重用,
导致一些判断的时候出了不想出现的结果,
毕竟原值已经被更改了。

这是一个问题,我下次一定要仔细看看,多做题,争取养成0bug的习惯。

(又看了若干道,用数学方法来做的数不胜数,但是也看到了一些有趣的工具)

在\vb|cmath|中提供的\vb|std::ceil|和\vb|std::floor|,
它们可以把float类型的值转换为相近的类整型浮点数。
其中\vb|std::ceil|负责把数进行``上升'',
而\vb|std::floor|负责把数进行``下降''。

除此之外还有\vb|std::round|和\vb|std::trunc|。
都是对浮点数进行离散的函数。

我自己是自己写了一个函数来实现\vb|std::ceil|的功能,性能不知道会差多少,但应该没差多少。


\section{cf's 1288B problem}\timetx{2020-01-15 09:01:57.170873}

数学真香,我爱数学,数学爱我。

这道题的题目定义了一个函数,叫做\vb|conc|,实例如下:

\begin{lstlisting}[language=python]
>>> conc(314, 15926535)
31415926535
>>> conc(5741, 59635)
574159635
\end{lstlisting}

原文是,``\vb|conc(a, b)| is the concatenation of a and b.''
也就是说,\vb|conc|把a和b拼接起来得到一个新的数,
有关系式$$conc(a, b) = a\times 10^{\text{len of }b\text{ in base 10}} + b$$
目前看来为了得到b在base 10下的长度只能用算法试出来。

如果有关系式满足$$conc(a, b) = a\cdot b + a + b$$化简后则恒有:
$$10^{\text{len of }b\text{ in base 10}} = b + 1$$

也就是说如果要在$(1\ldots A)$和$(1\ldots B)$中找到所有符合条件的a,b,
只需要求出$10^{\text{len of }b\text{ in base 10}} = b + 1$的b,
然后乘以A(因为所有的a都是满足条件的),答案就出来了。

当然除了数学问题,还有其他的很多问题,
比如说:

\begin{lstlisting}[language=C++]
while(temp--)
    ;//change the value of value;
\end{lstlisting}

上面这个代码可能不会运行temp遍的,因为temp在头部和循环题中用的都是同一个内存空间。

\subsection{后记}
(看来别人的回答后\ldots\ldots )

我对不起我的数学老师!
我单知道在Python里面是简单直接用\vb|len(str(num))|来获取长度,获取在十进制下的长度,
我不知道其他的函数还有更数学化的\vb|log10|!
我真傻,真的!

(手动狗头)

