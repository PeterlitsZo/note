
\section{cf's 1288C problem}\timetx{2020-01-16 06:55:23.613204}

这道题的题意经分析之后可以知道,主要是为了求得单调递增数列的可能排列数。

它给出了一个数列的长度$len$和每个元素的可能取值$1\le a_i\le n$。
我设可能排列数为$f(len, range)$,其中$range$是$n-1+1$的值,$a_i$只能取整数。
比如说,$f(2, 2)$是${1, 1}$,${1, 2}$,${2, 2}$三的排列的排列数,
故值为3。

注意到,$f(len, range)$在一般情况下,可以分为两种方面考虑,
第一种是数组的第一个元素取到了$range$对应的第一个元素,
所以剩下数组剩下的元素长度减一,元素取值范围不变(因为不是严格单调),
总的排列数为$f(len-1, range)$。

另一种是数组的第一个元素没有取到$range$对应的第一个长度,
这个时候总的排列数为$f(len, range-1)$。

故$f(len, range)$满足下式:

\[f(l, r)=
\begin{cases}
    1, & \text{if } r=1, \\
    r, & \text{if } l=1,\\
    f(l-1, r) + f(l, r-1), & \text{if } l\ne 1 \text{ and } r\ne1.
\end{cases}\]

所以建立一个$len\times range$的二维数组,自底向上把数组填满到$f(len, range)$,
就可以求出答案了。

这个之外我还犯了一个小问题,\vb|10e9+7|不是$10^9+7$,
因为在代码中那个\vb|e|就已经有10的意思了,所以正确写法是\vb|1e9+7|,
昨天找这个找了好久欸,真是太笨了。

\subsection{关于排列组合}

看了其他人的做法,发现有不少人用的方法是通过排列组合,
也就是通过$f(len, range)=C_{len+range-1}^{range}$来解决问题的。
我能理解,但是如果我一开始没有想到,那这样才可以把
$f(len, range)=f(len-1, range) + f(len, range-1)$
解成我想要的$C_{len+range-1}^{range}$呢?


\section{XeTeX 的命令行乱码}\timetx{2020-01-17 06:30:12.541703}

发现在命令行里XeTeX 中的信息输出老是乱码,
乱码总是会怀疑到编码的问题上来。

仔细在网上找了找,用\vb|cmd|的命令\vb|chcp 65001|改变了它的输出编码,
顿时输出的东西就正常了
(\vb|chcp 65001|是把编码页改到UTF-8上面去)。

现在有一个问题,是XeLaTeX 编码格式本身就是输出的UTF-8格式,
还是调用它的\vb|subprocess|的输出格式是UTF-8?

而且,我认为程序和shell之间用的是编码后的二进制数据流进行交流的,
是这么一回事吗?


\section{cf's 1285B problem}\timetx{2020-01-17 17:06:11.000355}

有点伤心,这道题我没有做出来。

这道题是为了求取最大的连续子串之和,令其为$f(list, start, end)$,
有三种可能情况。
第一种,数组串本身就是最优子串,其值为$sum(list.begin\to list.end)$。
第二种,最优子串不含有数组串最后一个元素,
所以其值为$f(list, start, end-1)$。
第三种和第二种类似,为$f(list, start+1, end)$。

可以注意到其中第二种和第三种有重复的地方,
比如$f(list, start+1, end-1)$就同时是第二和第三种情况的子串。

为了降低时间复杂度,可以使用动态规划。
我采用了自底向上的设计模式,所以每个子问题的解都依赖与之前的子子问题之解。
每个子问题都用了一块内存来存储其的最大的连续子串之和。
(我甚至还记录了它们对应的$sum(list.begin\to list.end)$!)
这样,计算每个子问题的时间复杂度为$O(1)$,
一共的复杂度为$O(n^2)$。

结果过不了?我晕了。

接下来说一些这个题揭露出来的问题。

\subsection{segmentation fault}
在编译后运行时直接跳出了这个错误,并提示发生了段错误。

它的意思是访问了不应该访问的地址,操作系统及时阻止了(干得好),
并抛出了错误原因。

\subsection{max}
关于求取最值,在之前有认识过求取容器中最值的标准库函数\vb|std::max_element|,
今天要用到其他的函数方法。

在非容器中,使用可以使用、\vb|std::max|,
它的一个原型如下:
\begin{lstlisting}[language=C++]
template< class T >
const T& max( const T& a, const T& b );
\end{lstlisting}
而另一个原型则如下:
\begin{lstlisting}[language=C++]
template< class T >
T max( std::initializer_list<T> ilist );
\end{lstlisting}

这个函数可以表示两个数的最大值,也可以表示一个initializer\_list的最大值。

比如说为了取得三者之间的最大值,可以使用:
\begin{lstlisting}[language=C++]
std::max(std::max(x, y), z);
\end{lstlisting}
或者使用更简单的语法结构:
\begin{lstlisting}[language=C++]
std::max({x, y, z});
\end{lstlisting}

\subsection{accumulate}
求和。

从第一个迭代器开始,一直求到最后一个迭代器。
包含第一个迭代器的值,但不包含最后一个。

所以说要表达类似Python中的\vb|[:]|,可以使用\vb|v.begin()|和\vb|v.end()|。
如果要表达\vb|[a:]|,则可以使用\vb|v.begin()+a|,而\vb|end|不变。
但是如果要表达\vb|[a:b]|的话,则应该表达为:
\vb|v.begin()+a|,而结尾则变为\vb|v.begin()+b|,
效果是一样的,都取不到最后一个值。

\subsection{最后,这道题的答案}
我的想法是把Yasser的和算出来,然后把Adel的最优子串算出来。
两者相互比较,最后判断输出答案。

但是他们的答案却不是这样。

他们的想法是:

对于给定数组$\{a_1, a_2, \ldots, a_i, \ldots, a_n\}$,
全部的和明显是$S_A = sum(\{a_1, a_2, \ldots, a_n\})$。

如果开始的部分和$sum(\{a_1, a_2, \ldots, a_i\}) \le 0$,
则明显有$sum(\{a_{i+1}, \ldots, a_n\}) \ge S_A$。

同理,结尾部分存在该子列,或者开头和结尾都存在这种子列,
都很明显满足这个条件。

逆否命题也成立。故原条件可以推广到更容易求解的状态。

所以说只需要在前后处寻找是否有符合条件的子列。
时间复杂度从$O(n^2)$变成了$O(n)$,只是求不出来最优子列的值了,
只能判断有无最优子列。


\section{cf's 1285C problem}\timetx{2020-01-23 09:41:49.392911}

现在还在\vb|running on test 36|,都不知道发生了什么。

\subsection{关于题面}
首先题面说了有$LCM(a,b)=X$,其中``$LCM(a,b)$ is the smallest positive integer
that is divisible by both a and b''。
很明显,如果存在数$x$,有$x|a$,$x|b$的时候,也就有$LCM(a,b)=LCM(\frac{a}{x},b)$,
因为$LCM(a,b)=\frac{ab}{gcd(a,b)}$,而又有$LCM(\frac{a}{x},b)=\frac{ab}{gcd(a,xb)}$,
两式等价。

所以对于$a$,$b$,很容易可以找到互质的两个数$c$,$d$,使得$LCM(a,b)=LCM(c,d)=cd$,
这时$c$,$d$最小,所以$max(c,d)$相对最小。

现在给定$N=cd$,不妨假定$c\le d$,则有$c\in [1,\lfloor\sqrt{N}\rfloor]$,
此时不难看出$N(\text{mod }c)=0$,也就是$c|N$,
且$gcd(c,d)=1$,不然得到的数为$\frac{cd}{x}$,而不为$cd$。

迭代一次,寻找满足条件的$c$的最大值,答案也就出来了。

\subsection{其他答案}
有一点点改进的空间,那就是逆向寻找,找到一个值就可以输出了。

