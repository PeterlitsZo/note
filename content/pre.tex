% 这是用来提前加载的预处理格式。
%%%%%%%%%%%%%%%%%%%%%%%%%%%%%%%%%%%%%%%%%%%%%%%%%%%%%%%%%%%%%%%%%%%%%%%%%%%%%%%

% 使用的宏包
%%%%%%%%%%%%%%%%%%%%%%%%%%%%%%%%%%%%%%%%%%%%%%%%%%%%%%%%%%%%%%%%%%%%%%%%%%%%%%%
\usepackage{amsmath}
\usepackage{amssymb}
\usepackage[stable]{footmisc}
\usepackage{transparent}
\usepackage{mathtools}
\usepackage{lipsum}
\usepackage{booktabs}
\usepackage{xcolor}
\usepackage{epigraph}

% 编程环境
%%%%%%%%%%%%%%%%%%%%%%%%%%%%%%%%%%%%%%%%%%%%%%%%%%%%%%%%%%%%%%%%%%%%%%%%%%%%%%%
\lstset{
    mathescape,
    morecomment=[l]{//},
    morecomment=[l]{\#},
}

% 过长的url
%%%%%%%%%%%%%%%%%%%%%%%%%%%%%%%%%%%%%%%%%%%%%%%%%%%%%%%%%%%%%%%%%%%%%%%%%%%%%%%
\g@addto@macro{\UrlBreaks}{\UrlOrds}

% 自定义的 marco
%%%%%%%%%%%%%%%%%%%%%%%%%%%%%%%%%%%%%%%%%%%%%%%%%%%%%%%%%%%%%%%%%%%%%%%%%%%%%%%

\newenvironment{epi}{%
    \begingroup
    \newcommand{\Text}[1]{\def\epiText{##1}}%
    \newcommand{\Source}[1]{\def\epiSource{##1}}%
    }{%
    \edef\Width{\dimexpr\linewidth - 2\parindent}
    \setlength{\epigraphwidth}{\Width}%
    %\setlength{\epigraphrule}{0pt}%
    \epigraph{\epiText}{\epiSource}%
    \endgroup
    }

% 给 old 的时间信息格式

\def\t#1|{\text{ \tt #1}}
\def\T#1|{\text{ #1}}
\def\*#1|{\text{#1}}
\def\&#1|{\text{\tt #1}}

\def\timetx#1{%
    \par\noindent\emph{\pltgray\small #1}\smallskip
    }

% 简短的vb,给 2020/6 使用。
\def\vb{\verb}

% 不被污染的导入
\def\inp#1{\begingroup\input#1\endgroup}

% 重点。
\def\impt#1{\par\smallskip\noindent{\bf #1}\par}
\def\Impt#1{\par\smallskip\noindent{\bf #1}}

% 示例
\newbox\ExpleTotal
\newbox\ExplePart

\long\def\ExpleSplitable#1{%
    \setbox\ExpleTotal = \vbox\bgroup%
        \hsize = \SubSep%
        #1%
    \egroup%
    \ExpleSplitableMainLoop}

\def\ExpleSplitableMainLoop{
    \dimen255 = \dimexpr \pagegoal - \pagetotal - \pageshrink \relax
    \ifdim \dimen255 < 0pt%
        \null
        \ExpleSplitableMainLoop
    \else\ifdim \ht\ExpleTotal < \dimen255%
        \indent\hbox to \MainSep{%
            \hfil\Rule\hfil\hfil%
            \vbox{\unvbox\ExpleTotal}%
        }\par%
    \else%
        \setbox \ExplePart = \vsplit \ExpleTotal to \dimen255%
        \indent\hbox to \MainSep{%
            \hfil\Rule\hfil\hfil%
            \vbox{\unvbox\ExplePart}%
        }\par%
        \eject%
        \ExpleSplitableMainLoop%
    \fi\fi}

\newenvironment{Exple}[1]{\begingroup%
    % the begining definition
    \edef\MainSep{\dimexpr\textwidth - \parindent}%
    \edef\SubSep{\dimexpr\textwidth - \parindent - 1.5em}%
    \def\Rule{\vrule width 1pt}%
    \linewidth = \SubSep
    % the main part
    \par\smallskip
    {\bf #1:}\par
    \setbox\ExpleTotal = \vbox\bgroup
        \hsize = \SubSep
        \vspace*{0.5em}%
}{
    \egroup
    \ExpleSplitableMainLoop
    \endgroup
}

\long\def\exple#1#2{\begingroup
    % the begining definition
    \edef\MainSep{\dimexpr\textwidth - \parindent}%
    \edef\SubSep{\dimexpr\textwidth - \parindent - 1.5em}%
    \def\Rule{\vrule width 1pt}%
    % the main part
    \par\smallskip
    {\bf 示例:}\par
    \ExpleSplitable{#1}\par
    {\bf 解答:}\par
    \ExpleSplitable{#2}\par
    \smallskip\par
    \endgroup}

% 源信息
%%%%%%%%%%%%%%%%%%%%%%%%%%%%%%%%%%%%%%%%%%%%%%%%%%%%%%%%%%%%%%%%%%%%%%%%%%%%%%%

\author{Peterlits Zo}
\title{Peter笔记}

