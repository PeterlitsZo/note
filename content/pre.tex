% 这是用来提前加载的预处理格式。
%%%%%%%%%%%%%%%%%%%%%%%%%%%%%%%%%%%%%%%%%%%%%%%%%%%%%%%%%%%%%%%%%%%%%%%%%%%%%%%

% 使用的宏包
%%%%%%%%%%%%%%%%%%%%%%%%%%%%%%%%%%%%%%%%%%%%%%%%%%%%%%%%%%%%%%%%%%%%%%%%%%%%%%%
\usepackage{amsmath}
\usepackage{amssymb}
\usepackage[stable]{footmisc}
\usepackage{transparent}
\usepackage{mathtools}
\usepackage{lipsum}

% 自定义的 marco
%%%%%%%%%%%%%%%%%%%%%%%%%%%%%%%%%%%%%%%%%%%%%%%%%%%%%%%%%%%%%%%%%%%%%%%%%%%%%%%

% 给 old 的时间信息格式

\def\timetx#1{%
    \par\noindent\emph{\pltgray\small #1}\smallskip
    }

% 简短的vb,给 2020/6 使用。
\def\vb{\verb}

% 类似公式的变量定义。
\def\ttva|#1|{ \verb|#1| }

% 不被污染的导入
\def\inp#1{\begingroup\input#1\endgroup}

% 重点。
\def\impt#1{\par\smallskip\noindent{\bf #1}\par}

% 示例
\newbox\ExpleTotal
\newbox\ExplePart

\long\def\ExpleSplitable#1{%
    \setbox \ExpleTotal = \vbox{%
        \hsize = \SubSep%
        #1%
    }%
    \ExpleSplitableMainLoop}

\def\ExpleSplitableMainLoop{\null
    \dimen255 = \dimexpr \pagegoal - \pagetotal - \pageshrink \relax%
    \ifdim \dimen255 < 0pt%
        \eject
        \ExpleSplitableMainLoop
    \else\ifdim \ht\ExpleTotal < \dimen255%
        \indent\hbox to \MainSep{%
            \hfil\Rule\hfil\hfil%
            \box\ExpleTotal%
        }\par%
    \else%
        \the\dimen255\par
        \setbox \ExplePart = \vsplit \ExpleTotal to \dimen255%
        \indent\hbox to \MainSep{%
            \hfil\Rule\hfil\hfil%
            \vbox{\unvbox\ExplePart}%
        }\par%
        \eject%
        \ExpleSplitableMainLoop%
    \fi\fi}

\long\def\exple#1#2{\begingroup
    % the begining definition
    \edef\MainSep{\dimexpr\textwidth - \parindent}%
    \edef\SubSep{\dimexpr\textwidth - \parindent - 1.5em}%
    \def\Rule{\vrule width 1pt}%
    % the main part
    \par\smallskip
    {\bf 示例:}\par
    \ExpleSplitable{#1\par}\par
    {\bf 解答:}\par
    \ExpleSplitable{#2\par}\par
    \smallskip\par
    \endgroup}

% 源信息
%%%%%%%%%%%%%%%%%%%%%%%%%%%%%%%%%%%%%%%%%%%%%%%%%%%%%%%%%%%%%%%%%%%%%%%%%%%%%%%

\author{Peterlits Zo}
\title{Peter笔记}
