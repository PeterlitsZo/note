
\section{GoLang的面向测试驱动开发}

\def\go{GoLang}
\def\ver|#1|{ \verb|#1| }

本笔记仅仅基于\go 的标准库\ver|testing|\footnote{官方文档在\url{https:
//golang.org/pkg/testing/}}。

该测试库,可以使用\ver|go test|来进行测试\footnote{记得之前说了要看Rust,
结果到现在才看来几页而已,不应该呀。}。

它提供了一个看起来非常amazing的接口,即,形如\ver|TestXxx(*testing.T)|的
函数。看起来和python的TDD模块特别像,但这是因为python的函数也是一个一等
对象,可以获取相应的名字。\go 身为一个静态编译性语言是如何做到这一点的,
我暂且不知。

\subsection{模仿和入门}

想要新建一个测试文档,首先需要新建一个以\ver|_test.go|结尾的文档\footnote{%
记得之前看\go 的一些相关源代码的时候,的确有很多以这个结尾的文件。没有想到
这不仅仅是一个编程习惯,原来它更是一种用来测试的必须的约定。}。

最简单的测试函数如下:
\begin{lstlisting}
func TestAbs(t *testing.T) {
    got := Abs(-1)
    if got != 1 {
        t.Errorf("Abs(-1) = %d; want 1", got)
    }
}
\end{lstlisting}

在测试函数中,\ver|Error|,\ver|Fail|和一些相关的函数都会导致测试失败。

\subsection{更多}

其实和python差不多,但是还有一些\go 的进阶内容。

\subsubsection{Benchmarks}

形如\verb|BenchmarkXxx(b *testing.B)|的函数会被认为是一个\verb|benchmark|,
只有在\verb|go test|使用了\verb|-bench|标志的时候会被顺序运行。

它对于计算运算性能很有帮助。

\subsubsection{Examples}

使用\verb|Examples|会将标准输出和注释相比较。

\subsubsection{Skipping}

\verb|*testing.T|支持跳过测试函数。

\subsubsection{Subtests and Sub-benchmarks}

支持运行子命令(通过一个函数对象)或者子板凳。

\subsubsection{Main}

有的时候,一个主入口也是很有帮助的。

%%%%%%%%%%%%%%%%%%%%%%%%%%%%%%%%%%%%%%%%%%%%%%%%%%%%%%%%%%%%%%%%%%%%%

\section{\LaTeX{}中可断开的vbox}

大家都知道\LaTeX{}中的盒子是不可被分页算法分割的,除非把一个盒子变成两个
分别的盒子\footnote{本节参考:\url{https://tex.stackexchange.com/quest
ions/20901/breakable-vboxes}}。所以一个,递归的、一次断裂的\verb|vbox|的
重要性就越来越显著了。

%%%%%%%%%%%%%%%%%%%%%%%%%%%%%%%%%%%%%%%%%%%%%%%%%%%%%%%%%%%%%%%%%%%%%

\section{\LaTeX{}中的宏和超长宏}

在我定义本文档的例子命令的时候,出现了一个意想不到的错误:
\begin{lstlisting}
Paragraph ended before \mymacro complete.
\end{lstlisting}

经过反复试错之后,我发现了在\verb|#n|中对应的组中如果出现了自动或者手动
生成的\verb|\par|命令之后才会出现这种错误。看来在\verb|\def|中不支持分段。

最后在Stack Overflow中找到了解决方案:使用\verb|\long\def|来代替\verb|\def|。
这是因为前面一个宏才会定义出超长宏,而后者只是短宏,不支持组中含有%
\verb|\par|命令(虽然不知道是怎么检测到的)。

%%%%%%%%%%%%%%%%%%%%%%%%%%%%%%%%%%%%%%%%%%%%%%%%%%%%%%%%%%%%%%%%%%%%%

\section{CF1374E1题解}

我感到很挫败,没有想到100题计划的第一题都这么出乎我的意料。

\subsection{贪心算法}

没有看懂源代码,但是评论区中有人说和贪心算法有关。所以就先预习一下贪心算法。

贪心算法,总的来说,就是只要每次都选出最优解,那么他就一定是最优解。而相对
而言,DP问题就是子问题不一定是最优解,但是可以让其成为最优解。这个题目我是
有想过用动态规划的。但是维护什么书借过什么没有的话,对我来说还是太困难一点了。

下面是IO网站说是贪心算法的一个例题。

\exple{%
    在一个月黑风高的暴风雨夜,Farmer John 的牛棚的屋顶、门被吹飞了。
    好在许多牛正在度假,所以牛棚没有住满。
    
    牛棚一个紧挨着另一个被排成一行,牛就住在里面过夜。有些牛棚里有
    牛,有些没有。所有的牛棚有相同的宽度。 
    
    自门遗失以后,Farmer John必须尽快在牛棚之前竖立起新的木板。他的
    新木材供应商将会供应他任何他想要的长度,但是吝啬的供应商只能提
    供有限数目的木板。Farmer John想将他购买的木板总长度减到最少。
    
    给出$m,s,c$,表示木板最大的数目、牛棚的总数、牛的总数;以及每头
    牛所在牛棚的编号,请算出拦住所有有牛的牛棚所需木板的最小总长度。
  
    \impt{输入格式}
    一行三个整数$m,s,c$,意义如题目描述。接下来$c$行,每行包
    含一个整数,表示牛所占的牛棚的编号。

    \impt{输出格式}
    输出一行一个整数,表示所需木板的最小总长度。
    }{%
    假设一开始只用一块,那么就是从头到尾一整块,这样虽然满足了要求,
    但是很明显$1\le m$,不够划算。

    那么从这个为起点,那么要选择从什么地方断开就最好了。

    既然要断开的话,那么就要选择在中间最大的地方、在两端断开。
    
    }

