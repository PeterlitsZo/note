
\section{GoLang的面向测试驱动开发}

\def\go{GoLang}
\def\ver|#1|{ \verb|#1| }

本笔记仅仅基于\go 的标准库\ver|testing|\footnote{官方文档在\url{https:%
//golang.org/pkg/testing/}}。

该测试库,可以使用\ver|go test|来进行测试\footnote{记得之前说了要看Rust,
结果到现在才看来几页而已,不应该呀。}。

它提供了一个看起来非常amazing的接口,即,形如\ver|TestXxx(*testing.T)|的
函数。看起来和python的TDD模块特别像,但这是因为python的函数也是一个一等
对象,可以获取相应的名字。\go 身为一个静态编译性语言是如何做到这一点的,
我暂且不知。

\subsection{模仿和入门}

想要新建一个测试文档,首先需要新建一个以\ver|_test.go|结尾的文档\footnote{%
记得之前看\go 的一些相关源代码的时候,的确有很多以这个结尾的文件。没有想到
这不仅仅是一个编程习惯,原来它更是一种用来测试的必须的约定。}。

最简单的测试函数如下:
\begin{lstlisting}
func TestAbs(t *testing.T) {
    got := Abs(-1)
    if got != 1 {
        t.Errorf("Abs(-1) = %d; want 1", got)
    }
}
\end{lstlisting}

在测试函数中,\ver|Error|,\ver|Fail|和一些相关的函数都会导致测试失败。

\subsection{更多}

其实和python差不多,但是还有一些\go 的进阶内容。

\subsubsection{Benchmarks}

形如\verb|BenchmarkXxx(b *testing.B)|的函数会被认为是一个\verb|benchmark|,
只有在\verb|go test|使用了\verb|-bench|标志的时候会被顺序运行。

它对于计算运算性能很有帮助。

\subsubsection{Examples}

使用\verb|Examples|会将标准输出和注释相比较。

\subsubsection{Skipping}

\verb|*testing.T|支持跳过测试函数。

\subsubsection{Subtests and Sub-benchmarks}

支持运行子命令(通过一个函数对象)或者子板凳。

\subsubsection{Main}

有的时候,一个主入口也是很有帮助的。

%%%%%%%%%%%%%%%%%%%%%%%%%%%%%%%%%%%%%%%%%%%%%%%%%%%%%%%%%%%%%%%%%%%%%

\section{\LaTeX{}中可断开的vbox}

大家都知道\LaTeX{}中的盒子是不可被分页算法分割的,除非把一个盒子变成两个
分别的盒子\footnote{本节参考:\url{https://tex.stackexchange.com/quest%
ions/20901/breakable-vboxes}}。所以一个,递归的、一次断裂的 \verb|vbox| 的
重要性就越来越显著了。

\subsection{vsplit命令}

\def\num{ \verb|<number>| }\def\dim{ \verb|<dimen>| }
命令\verb|\vsplit <number> to <dimen>|可以把\num 对应和盒子切割成两个部
分,第一部分的会输出而第二部分的则会保存到\num 对应的盒子。(注,\num 可
能之后会变成空盒子。

如:
\begin{lstlisting}
\setbox 20 = \vsplit 30 to 7in
\end{lstlisting}

\subsection{pagegoal和pagetotal命令}

背景知识: \verb|\vsize| 是定义单个页面的主方框的高度, \verb|\hsize| 则是设定
主方框的宽度\footnote{但是奇怪的是,我好像没有办法使用 \verb|\vsize| 来设定
之后页面的高度,相反,使用 \verb|\pdfpageheigtht| 倒是很有帮助。见页面\url{%
https://tex.stackexchange.com/questions/299005/automatic-page-size-to-fit%
-arbitrary-content}}。

命令 \verb|\pagetotal| 提供了当前页面的累计高度。而 \verb|\pagegoal| 则指明了
当前页面所需要的高度。 \verb|\pagegoal| 在一个页面开始时会检查并赋值自身为%
 \verb|\vsize| 。

\subsection{示例}

我们打算分割段落,所以我们打算用 \verb|\splitable| 来制作宏命令。
\begin{lstlisting}
\newbox\splitablebox
\def\splitable#1{%
    \setbox \splitablebox = \vbox{#1}%
    \box \splitable
}
\end{lstlisting}

目前为止还是中规中矩的一个不能分段的 \verb|\vbox| ,为了实现分段我们也把
一个 \verb|box| 分为两个 \verb|box| ,分别是 \verb|\totalbox| 和 \verb|\partialbox| %
。
\begin{lstlisting}
\newbox\splitableTotalbox
\newbox\splitablePartialBox
\def\splitable#1{%
    \setbox \splitableTotalbox = \vbox{%
        \advance\hsize by -2\fboxsep
        #1%
    }%
    \ifvoid \splitableTotalbox \else \splitableMainLoop \fi
}
\def\splitableMainLoop{%
    % \pageshrink 是目前页面的所有的胶水。
    \dimen255 = \dimexpr \pagegoal - \pagetotal - \pageshrink - 2\fboxsep \relax
    \ifdim\ht\splitableTotalbox < \dimen255
        \noindent\fbox{\box \splitableTotalbox}%
    \else
        \setbox \splitablePartialBox = \vsplit \splitableTotalbox to \dimen255
        \noindent\fbox{\box \splitablePartialBox}%
        \eject % eject 用来强制分页
        \splitableMainLoop
    \fi
}
\splitable{\longtext\longtext\longtext\longtext\longtext\longtext}
\end{lstlisting}

太棒了,我就是分割 \verb|vbox| 大师!(话说,有的时候需要加上 \verb|\null| ,
我也不知道为什么,总之,用递归来实现循环,总感觉怪怪的)。

有的时候, \verb|\dimen255| 甚至可能是个负数!这个太可怕了,请务必要试着
先判断一下。

%%%%%%%%%%%%%%%%%%%%%%%%%%%%%%%%%%%%%%%%%%%%%%%%%%%%%%%%%%%%%%%%%%%%%

\section{\LaTeX{}中的宏和超长宏}

在我定义本文档的例子命令的时候,出现了一个意想不到的错误:
\begin{lstlisting}
Paragraph ended before \mymacro complete.
\end{lstlisting}

经过反复试错之后,我发现了在\verb|#n|中对应的组中如果出现了自动或者手动
生成的\verb|\par|命令之后才会出现这种错误。看来在\verb|\def|中不支持分段。

最后在Stack Overflow中找到了解决方案:使用\verb|\long\def|来代替\verb|\def|。
这是因为前面一个宏才会定义出超长宏,而后者只是短宏,不支持组中含有%
\verb|\par|命令(虽然不知道是怎么检测到的)。

%%%%%%%%%%%%%%%%%%%%%%%%%%%%%%%%%%%%%%%%%%%%%%%%%%%%%%%%%%%%%%%%%%%%%

\section[CF-1374E1题解]{CF-1374E1题解\footnote{更多信息参见:\url{htt%
ps://codeforces.ml/problemset/problem/1374/E1}。}}

我感到很挫败,没有想到100题计划的第一题都这么出乎我的意料。

\subsection{题干}

Summer vacation has started so Alice and Bob want to play and joy, but... Their mom doesn't think so. She says that they have to read some amount of books before all entertainments. Alice and Bob will read each book {\bf together} to end this exercise faster.

There are $n$ books in the family library. The $i$-th book is described by three integers: $t_i$ — the amount of time Alice and Bob need to spend to read it, $a_i$ (equals $1$ if Alice likes the $i$-th book and $0$ if not), and $b_i$ (equals $1$ if Bob likes the $i$-th book and $0$ if not).

So they need to choose some books from the given $n$ books in such a way that:

\begin{enumerate}
    \item Alice likes {\bf at least} $k$ books from the chosen set and Bob likes {\bf at least} $k$ books from the chosen set;

    \item the total reading time of these books is {\bf minimized} (they are children and want to play and joy as soon a possible).
\end{enumerate}

The set they choose is {\bf the same} for both Alice an Bob (it's shared between them) and they read all books {\bf together}, so the total reading time is the sum of $t_i$ over all books that are in the chosen set.

Your task is to help them and find any suitable set of books or determine that it is impossible to find such a set.

\impt{Input}
The first line of the input contains two integers $n$ and $k$ ($1\le k\le n\le 2\cdot 10^5$).

The next $n$ lines contain descriptions of books, one description per line: the $i$-th line contains three integers $t_i$, $a_i$ and $b_i$ ($1\le t_i\le 10^4$, $0\le a_i$, $b_i\le 1$), where:

\begin{enumerate}
\item $t_i$ --- the amount of time required for reading the $i$-th book;
\item $a_i$ equals $1$ if Alice likes the $i$-th book and $0$ otherwise;
\item $b_i$ equals $1$ if Bob likes the $i$-th book and $0$ otherwise.
\end{enumerate}

\impt{Output}
If there is no solution, print only one integer -1. Otherwise print one integer T — the minimum total reading time of the suitable set of books.

\subsection{贪心算法}

没有看懂源代码,但是评论区中有人说和贪心算法有关。所以就先预习一下贪心算法。

贪心算法,总的来说,就是只要每次都选出最优解,那么他就一定是最优解。而相对
而言,DP问题就是子问题不一定是最优解,但是可以让其成为最优解。换句话说,局部最优
不一定是全局最优。但是对于贪心来说恰好就是局部最优意味着全局最优。这个题目我是
有想过用动态规划的。但是维护什么书借过什么没有的话,对我来说还是太困难一点了。

贪心算法的效率比用DP来得快很多。

下面是IO网站说是贪心算法的一个例题。

\exple{%
    在一个月黑风高的暴风雨夜,Farmer John 的牛棚的屋顶、门被吹飞了。
    好在许多牛正在度假,所以牛棚没有住满。
    
    牛棚一个紧挨着另一个被排成一行,牛就住在里面过夜。有些牛棚里有
    牛,有些没有。所有的牛棚有相同的宽度。 
    
    自门遗失以后,Farmer John必须尽快在牛棚之前竖立起新的木板。他的
    新木材供应商将会供应他任何他想要的长度,但是吝啬的供应商只能提
    供有限数目的木板。Farmer John想将他购买的木板总长度减到最少。
    
    给出$m,s,c$,表示木板最大的数目、牛棚的总数、牛的总数;以及每头
    牛所在牛棚的编号,请算出拦住所有有牛的牛棚所需木板的最小总长度。
  
    \impt{输入格式}
    一行三个整数$m,s,c$,意义如题目描述。接下来$c$行,每行包
    含一个整数,表示牛所占的牛棚的编号。

    \impt{输出格式}
    输出一行一个整数,表示所需木板的最小总长度。
    }{%
    假设一开始只用一块,那么就是从头到尾一整块,这样虽然满足了要求,
    但是很明显$1\le m$,不够划算。

    那么从这个为起点,那么要选择从什么地方断开就最好了。

    既然要断开的话,那么就要选择在中间最大的地方、在两端断开。

    这个真的是贪心算法吗,emmm。
    }

我感觉,贪心算法就是子问题最优总问题就最优。有的时候这个贪心还需要先排序。(不
过我听说使用最大堆或者最小堆的话会比较快一点)。

\subsection{题解}

我看了看别人的题解,感觉不是很难,先上伪代码:
\begin{lstlisting}
[]int all, aonly, bonly

switch cin >> t >> a >> b; (a, b) {
case (1, 1):
    all.append(t)
case (1, 0):
    aonly.append(t)
case (0, 1):
    bonly.append(t)
}

aonly.sort()
bonly.sort()
for i in min(aonly.len, bonly.len) {
    all.append(aonly[i] + bonly[i])
}
all.sort()

if all.len < k {
    cout << -1
} else {
    sum = 0
    for i in range(k) {
        sum += all[i]
    }
    cout << sum
}
\end{lstlisting}

主要来说,就是Alice和Bob两个人,按照贪心来说肯定是要时间越少越好,但是有一点不知
所措,那就是这个有两个属性:Alice是否感兴趣和Bob是否感兴趣。我想了一下,Alice感
兴趣,Bob感兴趣和都感兴趣的书并不是泾渭分明的,它们可能互相交叉,后来我就不知道
该怎么做了。

后来想了一下,看了看题解,因为Alice和Bob需要的都是$k$本感兴趣的书,所以读一本感兴趣
的书和读两本各自感兴趣的书在理论上相等。

所以我们可以先排序,排完序之后就可以把两本书合并成一本。然后把所有处理完后剩下的书
排序输出即可。

不过这有点巧,如果他们感兴趣的书的个数不相同呢?

