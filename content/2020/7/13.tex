
\section{在WSL上安装MySQL}

试着在WSL 1上安装MySQL来学习的,但是比我想象得要麻烦很多。有很多历史故事和特殊安
装方法,所以还是写一篇笔记比较好。

\subsection{一点小知识}

MySQL虽然还是开源的,但是因为它被一家商业公司给购买了,所以为了防止以后出于公司经
营角度而闭源,所以社区搞了一个分支叫做Mariadb来避免这种情况。Mariadb目前是百分百
的支持MySQL。

\subsection[开始安装]{开始安装\footnote{更多详情请参见:\url{https://medium.com%
/@harshityadav95/installing-mysql-in-ubuntu-linux-windows-subsystem-for-linux-f%
rom-scratch-d5771a4a2496}。}}

第一步,为了防止之前安装造成的影响,需要把之前的安装都删掉:
\begin{lstlisting}
sudo apt-get remove --purge *mysql*
sudo rm -rf /etc/mysql /var/lib/mysql
sudo apt-get remove --purge *mariadb*
\end{lstlisting}

\subsection{意外情况}

虽然是按照教程来一步一步走的,但是还是出现了一些意外情况。

\subsubsection{Unable to locate package ...}

在使用 \verb|apt-get remove| 的时候,出现了提示 \verb|Unable to locate package|%
。这让我大感疑惑:一般而言,这是在在安装的时候,找不大软件和它对应的来源的时候才
对抛出的错误,咋删除还需要呢?

一般而言在安装的时候出现这种错误的时候,就需要在 \verb|/etc/apt/sources.list| 中


%%%%%%%%%%%%%%%%%%%%%%%%%%%%%%%%%%%%%%%%%%%%%%%%%%%%%%%%%%%%%%%%%%%%%%%%%%%%%%%

\section{}

