
\section{在WSL上安装MySQL}

试着在WSL 1上安装MySQL来学习的,但是比我想象得要麻烦很多。有很多历史故事和特殊安
装方法,所以还是写一篇笔记比较好。

\subsection{一点小知识}

1.
MySQL虽然还是开源的,但是因为它被一家商业公司给购买了,所以为了防止以后出于公司经
营角度而闭源,所以社区搞了一个分支叫做Mariadb来避免这种情况。Mariadb目前是百分百
的支持MySQL。

2.
Socket file,是一种特殊的文件类型,它和TCP/IP的socket非常相似,提供了进程内网络的
保护。看起来,它好像能够在文件中传输数据。

\subsection[开始安装]{开始安装\footnote{更多详情请参见:\url{https://medium.com%
/@harshityadav95/installing-mysql-in-ubuntu-linux-windows-subsystem-for-linux-f%
rom-scratch-d5771a4a2496}。}}

第一步,为了防止之前安装造成的影响,需要把之前的安装都删掉:
\begin{lstlisting}
sudo apt-get remove --purge '*mysql*'
sudo rm -rf /etc/mysql /var/lib/mysql
sudo apt-get remove --purge '*mariadb*'
sudo apt-get autoremove
sudo apt-get autoclean
\end{lstlisting}

使用这个来进行检查:
\begin{lstlisting}
dpkg -l | grep mariadb
dpkg -l | grep mysql
\end{lstlisting}

之后修复所有的坏掉的软件:
\begin{lstlisting}
sudo apt-get install -f
\end{lstlisting}

然后开始升级软件源信息,并且升级所有的软件:
\begin{lstlisting}
sudo apt update
sudo apt upgrade
\end{lstlisting}

最后可以使用两个简单的命令来进行安装和配置:
\begin{lstlisting}
sudo apt install mysql-server
sudo mysql_secure_installation
\end{lstlisting}

\subsection{意外情况}

虽然是按照教程来一步一步走的,但是还是出现了一些意外情况。

\subsubsection[Unable to locate package ...]
              {Unable to locate package ...\footnote{更多信息请参见:\url{https%
://askubuntu.com/questions/1078404/unable-to-uninstall-package-with-error-unabl%
e-to-locate-package}。}}

在使用 \verb|apt-get remove| 的时候,出现了提示 \verb|Unable to locate package|%
。这让我大感疑惑:一般而言,这是在在安装的时候,找不到软件和它对应的来源的时候才
对此抛出的错误,咋删除还需要呢?

一般而言在安装的时候出现这种错误的时候,就需要在 \verb|/etc/apt/sources.list| 中
看看apt的源文件服务时候符合预期,然后使用一个 \verb|apt update| 就完事,但是如果
我是按照脚本或者直接安装的 \verb|.deb| 文件那怎么办呢?

\smallskip
附录:

我使用了 \verb|sudo apt remove ^mysql| 来代替 \verb|sudo apt remove *mysql*|。
然后就可以了。是我没有理解正则表达式吗???好奇怪。还有,如果使用 
\verb|sudo apt remove '*mysql*'| 也是可以运行良好的。是有什么特殊的处理吗?

\subsubsection{关于不存在一些文件产生的错误}

因为删除文件删除得太早了,所以导致删除不了软件。对此,我的解决方案是:re-install
and re-uninstall。重新安装之后不久没有了吗?我就是小天才。

\subsubsection{关于mysql-server}

我在安装的时候它说没有 \verb|mysql-server| 但是有 \verb|default-mysql-server|。
安装完了之后好像也没有什么不一样,我粗看了一下,好像它用的是MariaDB的二进制文件。

\subsubsection[关于无法连接socket文件]%
              {关于无法连接socket文件\footnote{更多信息请参见:\url{https://sta%
ckoverflow.com/questions/11990708/error-cant-connect-to-local-mysql-server-thro%
ugh-socket-var-run-mysqld-mysq}。}}

一般而言,指定的Socket文件配置在 \verb|/etc/my.cnf| 中,只需要在其中添加或者修改
为:
\begin{lstlisting}
socket=<$path\ to\ the\ socket\ file$>
\end{lstlisting}

而我们可以使用命令 \verb|sudo find / -type s 2> /dev/null| 来找到所有的Socket文件
。改成顺眼的就行。

但是很成功的失败了,于是试着用了一下 \verb|sudo service mysql start|。OK了。好奇怪
。

