
%%%%%%%%%%%%%%%%%%%%%%%%%%%%%%%%%%%%%%%%%%%%%%%%%%%%%%%%%%%

\section{HDU-6581题解}

今天打比赛,感觉脑袋晕晕呼呼的。状态的确有点点不太好。但是
好歹算是做出来了。

这道题是HDU-6581\footnote{原网址是\url{http://acm.hdu.edu.%
cn/showproblem.php?pid=6581}。}。

\subsection{题干}
Tom and Jerry are going on a vacation. They are now
driving on a one-way road and several cars are in
front of them. To be more specific, there are $n$ cars
in front of them. The $i$-th car has a length of $l_i$,
the head of it is $s_i$ from the stop-line, and its
maximum velocity is $v_i$. The car Tom and Jerry are
driving is $l_0$ in length, and $s_0$ from the stop-line,
with a maximum velocity of $v_0$.

The traffic light has a very long cycle. You can assume
that it is always green light. However, since the road
is too narrow, no car can get ahead of other cars. Even
if your speed can be greater than the car in front of
you, you still can only drive at the same speed as the
anterior car. But when not affected by the car ahead,
the driver will drive at the maximum speed. You can
assume that every driver here is very good at driving,
so that the distance of adjacent cars can be kept to be 0.

Though Tom and Jerry know that they can pass the
stop-line during green light, they still want to know
the minimum time they need to pass the stop-line. We
say a car passes the stop-line once the head of the car
passes it.

Please notice that even after a car passes the stop-line,
it still runs on the road, and cannot be overtaken.

\impt{Input}
This problem contains multiple test cases.

For each test case, the first line contains an integer
$n$ ($1\le n\le 10^5,\sum n\le 2\times 10^6$), the number
of cars.

The next three lines each contains $n+1$ integers, $l_i$,
$s_i$, $v_i$ ($1\le s_i,v_i,l_i\le 10^9$). It's guaranteed
that $s_i\ge s_{i+1}+l_{i+1},\forall i\in[0,n−1]$.

\impt{output}
for each test case, output one line containing the answer.
your answer will be accepted if its absolute or relative
error does not exceed $10^{−6}$.

formally, let your answer be $a$, and the jury's answer
is $b$. Your answer is considered correct if ${|a−b|\over
\max(1,|b|)}\le 10^{−6}$.

The answer is guaranteed to exist.

\subsection{我的见解}

我认为这个题目学长说应该用贪心算法来做,但是我其实不是很清楚
什么是贪心。

我认为对于第$i$-th的车子来说,它能完全过完所谓的stop-line的
所耗时间,其实取决于两点。第一点,就是它本身到停车线的时间,
也就是说应该为$s_i\over v_i$,另外一点,就是需要考虑到前面
一辆车从一开始出发到完全越过stop-line的时间了,也就是说应该
为$t_{i+1}(s_{i+1}+l_{i+1})=\max\left({s_{i+1}+l_{i+1}\over
v_{i+1}}, \ldots\right)$\footnote{这里$t_{i+1}$为一个函数。}。
只有从这两点出发都够到线的时候,才是真正的所花最短时间,我
将其定义为$t_i(s_i)$。其中参数$s_i$表示了研究路程的长度。

综上,有关系式$$t_i(s_i)=\max\left({s_i\over v_i}, t_{i+1}
(l_i+s_{i+1})\right)=\max\left({s_i\over v_i}, {l_{i+1}+
s_{i+1}\over v_{i+1}}, \ldots, {\sum_{g=i}^k l_g + s_k\over
v_k}, \ldots\right)$$以上可以刻画对于$s_i$路程的$i$-th的车子
而言,它的最短时间。

\def\ttsum{\text{\verb|sum|}}
不妨设置一个结构来表示车子,其中 \verb|l| 是车长, \verb|s| %
是车子到停车线的距离, \verb|v| 是车子的速度。这三个变量是由
输入提供的,而 \verb|sum| 则表示了$k$-th的车子从出发行进距离$
\sum_{g=i}^kl_g + s_k$比距离$s_k$所多走的距离,即有$\sum_{
g=i}^kl_g$,很明显有,$\ttsum_{i+1} = \ttsum_{i} + l_{i+1}$。
而${\ttsum_k+l_k\over v_k}={\sum_{g=i}^k l_g + s_k\over v_k}$

有伪代码:

\begin{lstlisting}
struct Car {
    int l, s, v, sum
}

vector<Car> cars
PER(i, n) cin >> cars[i].l
PER(i, n) cin >> cars[i].s
PER(i, n) cin >> cars[i].v

cars[0].sum = 0
FOR(i, 1, n) cars[i].sum = cars[i-1].sum + cars[i].l

double result = 0.0
PER(i, n) result = max(result, (cars[i].s + cars[i].sum) / cars[i].v)

cout << result
\end{lstlisting}

%%%%%%%%%%%%%%%%%%%%%%%%%%%%%%%%%%%%%%%%%%%%%%%%%%%%%%%%%%%
