
\section{关于C++的{\tt delete}}

C++的 \verb|delete| 可以用来干很多很多的事情,比如:
\begin{enumerate}
\item 释放 \verb|new| 声明的内存空间。
\item 释放 \verb|malloc| 声明的内存空间。
\item 和 \verb|NULL| 尝试友好相处。
\item 释放 \verb|new| 声明的数组内存空间。
\end{enumerate}

比如我不是很常用的释放数组内存空间如下:
\begin{lstlisting}
int* array = new int[10];
delete[] array;
\end{lstlisting}

或者和函数 \verb|malloc| 打配合:
\begin{lstlisting}
int* ptr = (int*)malloc(sizeof(int));
delete ptr;
\end{lstlisting}

最主要的是它能代替 \verb|free| 函数的就是:它是一个操作符,支持重定义,所以能很好
得处理用户自定义数据结构。

%%%%%%%%%%%%%%%%%%%%%%%%%%%%%%%%%%%%%%%%%%%%%%%%%%%%%%%%%%%%%%%%%%%%%%%%%%%%%%%

\section{C++的{\tt sizeof}关键字}

\verb|sizeof| 关键字有两种表达方式,一种是用来得到值对应类型所占用的内存长度,一
种则是直接得到类型所对应的内存空间。分别可以写为 \verb|sizeof(<$type$>)| 和 
\verb|sizeof <$expression$>|。

众所周知,计算机科学中的最小单位是 \verb|bit|,而 \verb|sizeof| 返回的单位是 
\verb|byte|,也是计算机科学中最小能利用的单位。一般而言,一个 \verb|byte| 的长度
为8个或者更多的 \verb|bits|。

对于数组来说我们不妨举一个例子:
\begin{lstlisting}
int a[10];
cout << sizeof a               << " "   // output: 40 (if sizeof(int) == 4)
     << sizeof(int[10])        << " "   // output: 40 (if sizeof(int) == 4)
     << (sizeof a / sizeof *a);         // output: 10 (whatever)
\end{lstlisting}

我们可以看到如果是一般的数组名的话,它的类型是 \verb|int[10]| 而不是 \verb|int*|,
但是在一些情况下(快给我变!)可以进行转换。比如说 \verb|*a| 的类型就是 
\verb|*&a[0]=a[0]| 了。

%%%%%%%%%%%%%%%%%%%%%%%%%%%%%%%%%%%%%%%%%%%%%%%%%%%%%%%%%%%%%%%%%%%%%%%%%%%%%%%

\section{关于JavaSricpt的压缩对象数组}

这个笔记是基于Stack Overflow的一个问题\footnote{更多信息请参见:\url{https://st%
ackoverflow.com/questions/62857540/what-is-the-simplest-solution-to-flat-an-jav%
ascript-array-of-objects-into-an-ob}。}。

虽然被关闭了,但是我觉得这个问题很有意思。

\subsection{引入问题}

存在一个array如下:
\begin{lstlisting}
a = [{x: 1}, {y: 2}]
\end{lstlisting}

我们需要的就是将所有的对象压缩成一个对象,即有:
\begin{lstlisting}
b = {x: 1, y: 2}
\end{lstlisting}

\subsection{一些简单知识}

作为一个优秀的动态类型语言,能够通过字符串这种值来控制属性自然是JavaScript必须的%
的内容,比如说:\verb|b['x'] == b.x|。换句话说,属性可以通过同名的字符串作为下标%
来得到。

另外我们可以array具有方法 \verb|forEach|来接受一个函数来进行遍历,惊喜的是,
\verb|Object.keys(<obj>)| 恰好可以返回对象的Keys的字符串列表。

\subsection{我的解}

基于上面的一些简单知识。我尝试解出如下:
\begin{lstlisting}
> a = [{x: 1}, {y: 2}]
> b = {}
> a.forEach((obj) => {
.....Object.keys(obj).forEach((key) => {
.......b[key] = obj[key]
.....})
...})
\end{lstlisting}

\subsection{其他的解}

\begin{lstlisting}
const object = array.reduce((target, item) => {
    return { ...target, ...item }
}, {})
\end{lstlisting}

其中three dots(\verb|...|)有以下几种翻译:
\begin{enumerate}
\item 和golang一样用来在函数头部中取多余的值,以一个列表的形式返回。
\item unpack,也是和golang一样,比如说
\begin{lstlisting}
const c = [...a, 12, ...b]
\end{lstlisting}

当然除了数组,对象也不在话下。
\begin{lstlisting}
const c = {...a, peter: 12, ...b}
\end{lstlisting}
\end{enumerate}
 
而array的方法 \verb|reduce| 可以用来依次的对两个元素进行操作,然后最终返回一个
nice的返回值。

%%%%%%%%%%%%%%%%%%%%%%%%%%%%%%%%%%%%%%%%%%%%%%%%%%%%%%%%%%%%%%%%%%%%%%%%%%%%%%%

\section{CF-1255-E1题解}

%%%%%%%%%%%%%%%%%%%%%%%%%%%%%%%%%%%%%%%%%%%%%%%%%%%%%%%%%%%%%%%%%%%%%%%%%%%%%%%
