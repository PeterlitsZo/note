
\section{C++容器}

C++的容器分为两种,一种是顺序容器,另外一种是关联容器。其中顺序容器的储存顺序是
按照编码执行顺序进行储存的,而关联容器则是根据值本身进行排序的。

比如说 \verb|map|,\verb|set| 和用 \verb|unordered| 或者 \verb|multi| 进行修饰
后的容器。关联容器的类型别名有三个,主要是 \verb|key_type|,\verb|value_type|%
,其中 \verb|key_type| 是用来支持它在内存中到底储存的,而 \verb|value_type| 则是
它储存的东西。

一般而言有序容器都是依靠 \verb|less| 来进行比较的,比如:
\begin{lstlisting}
template <class T> struct less {
    bool operator(const T& x, const T& y) const {return x < y;}
    typedef T first_argument_type;
    typedef T second_argument_type;
    typedef bool result_type;
};
\end{lstlisting}

当然这只是一个默认值,根据 \verb|set| 的定义:
\begin{lstlisting}
template <class T,                      // set::key_type/value_type
          class Compare = less<T>,      // set::key_compare/value_compare
          class Alloc = allocator<T>    // set::allocator_type
         > class set;
\end{lstlisting}

我们可以使用自定义的,能接受两个参数的函数指针或者自定义好的同样内置的模板:
\begin{lstlisting}
  equal_to
  not_equal_to
* greater
  greater_equal
  less_equal
\end{lstlisting}

这样我们能通过迭代器迭代来获得期望的顺序。不过注意的是,\verb|set| 迭代器不能用
来比较大小,这个时候应该使用 \verb|!=| 操作来进行迭代。

%%%%%%%%%%%%%%%%%%%%%%%%%%%%%%%%%%%%%%%%%%%%%%%%%%%%%%%%%%%%%%%%%%%%%%%%%%%%%%%

\section{关于Python的包}

在Python中一般用一个包含 \verb|__init__.py| 的文件的文件夹来表示一个包,在 %
\verb|__init__.py| 中会尝试先运行一些命令,比如我有一个文件夹如下:
\begin{lstlisting}
 .
  +- datedate\
  |   +- __init__.py
  |   +- main.py
  +- datedate.py
\end{lstlisting}

我可以在 \verb|__init__.py| 中指出 \verb|from .main import main|,然后我就可以在 %
\verb|datedate.py| 中使用 \verb|datedate.main| 而不是 \verb|datedate.main.main| 了
,换句话说,这是导入了命名空间集。

%%%%%%%%%%%%%%%%%%%%%%%%%%%%%%%%%%%%%%%%%%%%%%%%%%%%%%%%%%%%%%%%%%%%%%%%%%%%%%%

\section{LeetCode-178 题解}

因为SQL关于变量的支持好像不太好(我的天,如果支持轻松地使用变量的话,我会这么恼
火?),所以这道题还是屁颠屁颠地看别人的答案。心情不好。

\subsection{题干}

编写一个 SQL 查询来实现分数排名。

如果两个分数相同,则两个分数排名(Rank)相同。请注意,平分后的下一个名次应该是下一个连续的整数值。换句话说,名次之间不应该有“间隔”。
\begin{lstlisting}
+----+-------+
| Id | Score |
+----+-------+
| 1  | 3.50  |
| 2  | 3.65  |
| 3  | 4.00  |
| 4  | 3.85  |
| 5  | 4.00  |
| 6  | 3.65  |
+----+-------+
\end{lstlisting}

例如,根据上述给定的 Scores 表,你的查询应该返回(按分数从高到低排列):
\begin{lstlisting}
+-------+------+
| Score | Rank |
+-------+------+
| 4.00  | 1    |
| 4.00  | 1    |
| 3.85  | 2    |
| 3.65  | 3    |
| 3.65  | 3    |
| 3.50  | 4    |
+-------+------+
\end{lstlisting}

重要提示:对于 MySQL 解决方案,如果要转义用作列名的保留字,可以在关键字之前和之后使用撇号。例如 `Rank`

\subsection{预先部分}

1.
\verb|JOIN| 操作能够把来自两个表或多个表的行给结合起来。最基础的是笛卡尔联结,
如果把表1记为$\alpha$而把表2记为$\beta$,那么两表的最基础的联结就为笛卡尔积,
记为$\alpha\times\beta$。示例:
\begin{lstlisting}
mysql> SELECT * FROM a;
+----+-------+
| id | name  |
+----+-------+
|  1 | peter |
|  2 | alice |
+----+-------+
2 rows in set (0.001 sec)

mysql> SELECT * FROM b;
+------+---------------------+
| a_id | thing               |
+------+---------------------+
|    1 | a nice person maybe |
|    2 | a good gril         |
|    1 | good idea           |
|    1 | help                |
+------+---------------------+
4 rows in set (0.001 sec)

mysql> SELECT * FROM a, b;
+----+-------+------+---------------------+
| id | name  | a_id | thing               |
+----+-------+------+---------------------+
|  1 | peter |    1 | a nice person maybe |
|  2 | alice |    1 | a nice person maybe |
|  1 | peter |    2 | a good gril         |
|  2 | alice |    2 | a good gril         |
|  1 | peter |    1 | good idea           |
|  2 | alice |    1 | good idea           |
|  1 | peter |    1 | help                |
|  2 | alice |    1 | help                |
+----+-------+------+---------------------+
8 rows in set (0.001 sec)
\end{lstlisting}

对于这个可以采取 \verb|WHERE a.id = b.a_id| 来进一步筛选结果。

而内部联结则需要关键词 \verb|ON| 不过对于使用 \verb|WHERE| 的交叉连接而言,它
的效率会相对高一点点。示例:
\begin{lstlisting}
mysql> SELECT a.id, a.name, b.thing
    -> FROM a JOIN b ON a.id = b.a_id;
+----+-------+---------------------+
| id | name  | thing               |
+----+-------+---------------------+
|  1 | peter | a nice person maybe |
|  2 | alice | a good gril         |
|  1 | peter | good idea           |
|  1 | peter | help                |
+----+-------+---------------------+
4 rows in set (0.001 sec)
\end{lstlisting}

\subsection{题解}

首先是左边的部分,不是很难:
\begin{lstlisting}
SELECT a.Score as Score
FROM Scores AS a
ORDER BY a.Score DESC
\end{lstlisting}

然后是右边的部分:
\begin{lstlisting}
SELECT b.rank AS `Rank` FROM Scores JOIN
(
    SELECT p.score AS score, @rank := @rank+1 AS `rank` FROM
        (SELECT distinct(score) AS score FROM Scores ORDER BY score DESC) AS p,
        (SELECT @rank := 0) AS q
) AS b ON Scores.score = b.score
ORDER BY score DESC;
\end{lstlisting}

总的来说,就是在内部中搞了一个成绩与名次对应表。

不过我不是这么做的,我的想法如下:
\begin{lstlisting}
SELECT Score, `Rank` FROM (
    SELECT Score,
           @rank := IF(Score = @prev, @rank+0, @rank+1) AS `Rank`,
           @prev := Score
    FROM Scores, (SELECT @rank := 0, @prev := NULL) q
    RDER BY Score DESC
) q;
\end{lstlisting}

%%%%%%%%%%%%%%%%%%%%%%%%%%%%%%%%%%%%%%%%%%%%%%%%%%%%%%%%%%%%%%%%%%%%%%%%%%%%%%%

\section{MySQL笔记 \#2}

跳过了很多,直接看看第三节。

\subsection{数据值的类别}

数据有很多种种类,在MySQL中,又类型数值(如 \verb|48|,\verb|193.62|,或者 %
\verb|-2378E12|),字符串值(如 \verb|'peter'| 或者 \verb|'12345'|),日期值(
如 \verb|'CCDD-MM-DD'| 或者 \verb|'2011-06-17 12:30;43'|),布尔值和 \verb|NULL| %
值。

\subsection{{\tt IF}表达式}

\verb|IF| 表达式的语法如下:
\begin{lstlisting}
IF(expr1, expr2, expr3)
\end{lstlisting}

其中如果 \verb|expr1| 成立的话返回 \verb|expr2| 不然就返回 \verb|expr3|。

\subsection{{\tt :=}在{\tt SELECCT}}

和C/C++一样,使用了 \verb|:=| 之后,就会返回所赋的值。
不太舒服的是,初始化不能用 \verb|SET| 而只能在 \verb|FROM| 中进行初始化。

%%%%%%%%%%%%%%%%%%%%%%%%%%%%%%%%%%%%%%%%%%%%%%%%%%%%%%%%%%%%%%%%%%%%%%%%%%%%%%%
