
%%%%%%%%%%%%%%%%%%%%%%%%%%%%%%%%%%%%%%%%%%%%%%%%%%%%%%%%%%%

\section{默认小数位数}

\def\vb|#1|{ \verb|#1| }

在C++中的\vb|ostream|引导的流输出中,根据对浮点数类型的重载,本身的位数并不
是很多,所以被判定为答案错误,其实本质上是精度不够。

使用一些可控制符来对齐进行定义:
\begin{lstlisting}
#include<iomanip>
...
cout << setiosflags(ios::fixed);            //保证setprecision()是设置小数点后的位数。
cout << setprecision(2) << pi << endl;      //输出3.14
cout << pi << endl;                         //输出3.14
\end{lstlisting}

其中\vb|std::fixed|是用来保证浮点数使用定点输出,即fixed-point notation。
而\vb|setprecision|则是用来定义浮点数的位数的。

或者使用\vb|std::cout.precision(<int>)|来直接定义输出流的浮点数保留精度。

我不是很懂\vb|std::fixed|和\vb|setiosflags(std::fixed)|之间有什么不同,
但是在翻看了一些StackOverflow的回答,好像有很多都是一样的。

%%%%%%%%%%%%%%%%%%%%%%%%%%%%%%%%%%%%%%%%%%%%%%%%%%%%%%%%%%%

\section{数论 --- 关于被整除,和其扩展}

\subsection{一元的被整除}

整除,和不整除,是二元的,换言之,两个数之间,要不然有整除关系,
要不然没有,不存在介于整除和不整除的关系。所以说研究未被整除,
本质上也是在研究什么情况下会被整除。

\def\a{\text{\tt a}}
\def\b{\text{\tt b}}
\def\mod{\text{mod }}
\def\floor#1{\lfloor{#1}\rfloor}
\def\ceil#1{\lceil{#1}\rceil}
\def\lcm{\text{lcm}}

对两个数$\a$和$\b$而言,不妨假设$\a > \b$,那么当且仅当
$\a = \b\times k, k\in R$的情况下,两数之间存在整除关系。一般而言,
我们会使用$\a (\mod \b)$或者$a\%b$来判断它们之间的余数是否为零
来间接判断它们之间是否有整除关系。

现在将$\a$的范围从一个数,扩大到$[m, n]$,研究能被整除的个数\footnote{%
同时也是在研究不被整除的个数。假设$\a(\mod\b)=0$的$\a$有$t$个,那么
不被整除的$\a$的个数则自然为$n-m+1-t$}。根据前面的定义,两个符合条件
的$\a$最短距离则为$|\b\times k-\b\times (k-1)|=\b$,那么自然由于满足条件的%
$\a$的间隔性,所以其$\a$的个数几乎接近于$n - m + 1\over b$。

说几乎,但是并不是正确答案,因为$\a$的个数是离散的,对于不同的$m$,或者$n$,
即使上下限之间的距离一样,也可能会多一个$\a$或者少一个,在这种情况下$\a$的数量到底是
呈现一个什么样规律呢?

我注意到,当$m$和$n$恰好卡在某一个$\a$上的时候和未卡在$\a$上的时候,前后的
$\a$的数值相差为1。对$n > m$的$n$作为研究对象,$n\in[k\b, k\b + \b - 1]$的时候,
其$\a$的数值一样。所以我们不妨将其进行运算$\floor{n\over\b}=k$,同理对$m$进行运算,
即$\ceil{m\over\b} - 1$。两式相减有$\floor{n\over\b}-\ceil{m\over\b} + 1$,
其值就是对于一元的未被整除的$\a$的个数的值。

{\bf 示例:}%
对于$\a\in[3, 19]$,$\b=5$而言,不难看出来,$\floor{n\over\b}=
\floor{19\over 5}=3$,$\ceil{m\over\b}-1=\ceil{3\over 5}-1=0$,
两式相减为$3$,即5,10,15三个数字。

{\bf 分析易知:}%
其中$\floor{n\over\b}$代表了从0到n之间,有多少个符合条件的$\a$,而$\ceil{m\over\b}-1$%
表示了从0到m之间(但是不包含m)有多少个符合条件的$\a$。

\subsection{多元的被整除}

对于多元的未被整除,就是将$\b$扩展到了更高层次的、离散化的元组了。举个例子,
对于$[3, 18]$中可以被其中一个在$(2, 5, 7)$中的数整除的数$\a$而言,求符合条件
的$\a$的个数。

我们不妨将一元的被整除的函数抽象为一个函数$f(m, n, \b)=
\floor{n\over\b}-\ceil{m\over\b} + 1$,从直观的角度而言,我们可能会对例子
而言列出式子:$f(3, 18, 2) + f(3, 18, 5) + f(3, 18, 7)$,但是它错得也很容易理解,
如图\ref{pic:venn}所示,被2整除的数,可能同时也被5整除,甚至可能同时被7整除,
对于这个$\a$而言,它因为一个愚蠢的加法运算,用一个数搞出了有三个数的效果。
效果很棒,但很明显是错误的,是不够准确的。

\begin{figure}
    \centering
    \def\firstC{(0,0) circle (1.5cm)}
    \def\secondC{(60:2cm) circle (1.5cm)}
    \def\thirdC{(0:2cm) circle (1.5cm)}
    \begin{tikzpicture}
        \fill[black,opacity=0.2] \firstC;
        \fill[black,opacity=0.2] \secondC;
        \fill[black,opacity=0.2] \thirdC;
        \draw[color=black!50] \firstC;
        \draw[color=black!50] \secondC;
        \draw[color=black!50] \thirdC;
        \node at(0,0) {被2整除};
        \node at(60:2cm) {被5整除};
        \node at(0:2cm) {被7整除};
    \end{tikzpicture}
    \caption{对于$[m,n]$而言符合条件的$\a$的韦恩图,分别对应了未被3、5、7整除}
    \label{pic:venn}
\end{figure}

借鉴于概率学,很容易知道,因为两个公式$f(m,n,\alpha)$和$f(m,n,\beta)$之间重合的
部分,恰好是能够同时被被$\alpha$和$\beta$整除的数字。也就是说$\a=k\alpha\beta$,
换言之,中间的值为$f(m,n,\alpha\beta)$。和概率学同理,对于上面的示例的答案为:%
$g(x)=f(m,n,x);\ g(2)+g(3)+g(5)-g(2\times5)-g(2\times7)-g(5\times7)+g(2\times3\times7)$

推广有:对于范围$[m,n]$,存在$r$个数对于$(a_1, a_2, \ldots, a_n)$中至少一个数
而言有整除关系。令$$f(m, n, \b)=\floor{n\over\b}-\ceil{m\over\b} + 1$$其值易知,
由于一元被整除关系的推导,为在范围$[m, n]$之间能被$\b$整除的数$\a$的个数。那么对于上式
有:$$r=\smashoperator{\sum_{1\le i\le n}}f(m,n,a_i)-
        \smashoperator{\sum_{1\le i<j\le n}}f(m,n,\lcm(a_i, a_j))+
        \smashoperator{\sum_{1\le i<j<k \le n}}f(m,n,\lcm(a_i, a_j, a_k))-\ldots$$

%%%%%%%%%%%%%%%%%%%%%%%%%%%%%%%%%%%%%%%%%%%%%%%%%%%%%%%%%%%

\def\Cp{C/C++}

\section{\Cp 的宏定义}

\subsection{基本用法}

在\Cp 的宏定义语法中,最常见的就是将一个文本中字符串,或者
是文中中类似函数的结构扩展为另一个字符串。这些过程都发生在
编译之前,被称为预处理过程。
\begin{lstlisting}
#define IDENTIFIER TOKEN-STRING(opt)
#define IDENTIFIER(IDENTIFIER(opt), ..., IDENTIFIER(opt)) TOKEN-STRING(opt)
\end{lstlisting}

在预处理过程中,是可以通过\verb|#ifdef|或者\verb|#idndef|来
测试是否有该宏定义。这就是最基本的用法了。

值得注意和是,\Cp 中的宏定义替换只会替换一个\verb|token|而
不是其他的结构字符串。所以在\verb|string|或者在\verb|comment|%
中的字符串则是永远不会被替换的。

\subsection{长的宏定义}

使用长一点点的宏定义,有时候为了美观必须要断行,而一般的%
\verb|TOKEN-STRING|是无法断行的,为了断行第一个特殊字符\verb|\|
就出来了。如果\verb|\|的后面是一个\verb|newline|字符,那么
它会取消掉该\verb|newline|对它的不理影响,即,让该\verb|newline|
不再作为一个断行符号、一个\verb|#define|命令的结束符号。

\subsection{特殊的格式}

在第二种格式(即,类似函数式定义的时候),有一些特殊的用法,
比如\verb|##|,\verb|#|和\verb|#@|,分别是:形成一个原类似于
参数的IDENTIFIER之间的的连接(拼接),字符串化和字符化。

在我常用的过程中,我习惯使用以下的宏定义:
\begin{lstlisting}
#define FUCK(token) #token": " << token << "; "

int a = 114514, b = 314524;
cout << FUCK(a) << FUCK(b) << endl;
\end{lstlisting}

其中,\verb|FUCK(a)|会被扩展为\verb|"a"": " << a << "; "|,
其中根据定义两个字符串会粘合在一起,成为\verb|"a: "|,所以说
其相关的输出应该如下:
\begin{lstlisting}
a: 114514; b: 314524; %
\end{lstlisting}

%%%%%%%%%%%%%%%%%%%%%%%%%%%%%%%%%%%%%%%%%%%%%%%%%%%%%%%%%%%

\section{Python基础概念}

\def\pyth{\verb|python|}

理解\pyth 中的基本概念,才是学习\pyth 中需要注意的一步。

\subsection{对象}

一门简单的语言,必定是抽象程度足够高的语言,在\pyth 中,最基本的
概念,也同时是最重要的概念,就是对象了。而不同对象的区分就在于类型
(type)。当然作为一门足够高抽象的语言,类型也是一个对象。

\begin{lstlisting}
In [1]: # 数字1是对象吗?
In [2]: isinstance(1, object)
Out[2]: True

In [3]: # 太棒了,那他是什么类型的对象呢?很明显,1的类型是int(object中的一
   ...: # 种类型),int本身是一种类型没错吧,所以int的类型是type。type是int的
   ...: # 类型,所以type也是一种类型(一种可以调用的对象类型)。
In [4]: type(1), type(int), type(int)(int), type(type)
Out[4]: (int, type, type, type)

In [5]: # 看来是int类型的对象,int类型可以干什么呢?
In [6]: help(int)       # 使用help(1)也会有一样的结果哦。
class int(object)       # 看来int是一种object,所以说int对象也是一种对象
 |  int([x]) -> integer # 将其他东西括起来,能变成int对象的就会变。
 |  Convert a number or string to an integer, or return 0 if no arguments
 |  are given. If x is a number, return x.__int__(). ...
 |                      # int除了是一个类型,它本身也是可以调用的。
 |                      # 调用int,会把数字和字符串转换为整数。
 |                      # 原理上,是调用了x.__int__()。
 |  ...
 |  __add__(self, value, /)
 |      Return self+value.
 |                      # 原来是这样:每一个操作符,底层都是一个函数。
 |  ...

In [7]: # 看来我声明了一种peter类型,这种类型产生的对象,都对应着数字
   ...: # 31425926
In [8]: class peter(object):
   ...:     def __int__(self):
   ...:         return 31415926
In [9]: int(peter())
Out[9]: 31415926
In [9]: "peter() have the integer value: %d" % peter()
Out[9]: 'peter() have the integer value: 31415926'

In [10]: # 还记得之前说过对象的运算符操作底层是一个函数吗?
In [11]: "peter() have the integer value: %d".__mod__(peter())
Out[11]: 'peter() have the integer value: 31415926'

In [12]: # 看来的确如此,1 + 1和1.__add__(1)一样,"%d" % 1和"%d".__mod__(1)
    ...: # 自然也是一样的。这么看来,格式运算符也没有什么大不了的,它和1 + 1
    ...: # 一样简单,不过是把所有以%开头的,以字母结尾的字符区间替换为相应
    ...: # 的东西,比如以s结尾,就会让它以string形式输出,如果是d,就是以digit
    ...: # 形式输出。
\end{lstlisting}

\subsection{常见复合类型}

\pyth 中有常见的复合类型,为列表(list),元组(tuple),和词典
(dict),它们前两个都可以同时储存不同类型的对象\footnote{在底层
内存模型不一样的情况下,提供这种功能的语言并没有想象得那么多。毕竟
这种符合一般人高抽象层次的理想解决方案,在实现上都比较消耗内存。},
而词典则提供了一种\verb|key:value|的抽象数据结构。

\begin{lstlisting}
In [1]: # 太棒了,一个可以容纳各种各样的数据结构。无论是int类型的整数object,
   ...: # 还是string类型的字符串object,或者是float类型的浮点数object.
In [2]: [1, 'this', 1.3]
Out[2]: [1, 'this', 1.3]
\end{lstlisting}

\subsection{去序列化}

在\pyth 中有一种语法糖:它就是去序列化,本质上是元组的解序列化。

%%%%%%%%%%%%%%%%%%%%%%%%%%%%%%%%%%%%%%%%%%%%%%%%%%%%%%%%%%%

\section{if-idf排序}

和之前的布尔查询不同,if-idf杜绝了结果太多或者太少:它使用了
相关度来进行排序。通过对文档进行$[0,1]$之间的赋值,从而能够
更好的服务。

\subsection{关联矩阵}

关联矩阵可以用来描述一个文档。之前的数据结构,我用的是单词对应
它所对应的文档序号。我可以很轻松的对文档序号链表进行逻辑操作,
但是我无法刻画出这个文档的全貌。关联矩阵就是一个用来描述文档
而不是单词的一种数据结构。

\subsubsection{二值关联矩阵}

把每一篇文档都看成一个二维的向量,每一个坐标都代表了那个坐标
对应的单词是否出现\footnote{合理的想想,这样的话每一个文档的
向量长度都明显有点过长了,因为单词集很长啊!}。

\subsubsection{非二值关联矩阵}
\label{sec 非二值}

非二值关联矩阵,就是用词频(很明显,是一个整数)来描述文档,而
不是用布尔值来描述文档。

但是一般而言,相关度不会正比于词频。所以说,一般使用对数词频来
代替原始的词频:

\def\tif{\text{if }}

$$iw_{t,d} = \left\{\begin{array}{ll}
    1+\log_{10}tf_{t,f} & \tif\ tf_{t,d}>0  \\
    0 & otherwise \end{array} \right.$$

这就是单词t在文档d中的频数的对数频数。

\subsection{权重}

为了更好的获得搜索结果,我们应该要分清楚罕见词汇和不罕见词汇。如果
包含了一个罕见词汇的话,对它的权重自然更将应该比其他的大\footnote{%
从另外一个方面来看待的话,这是因为罕见词汇的信息量更大一些。越稀
少的单词所蕴含的信息量自然更大一些。}。

\subsubsection{文档频率\footnote{因为它的英文是Document frequency,
所以又可以被叫做df}}

在\ref{sec 非二值}中说过,可以使用一个词频向量来刻画一个文档。而
文档频率则是用来刻画单词的。它是用来刻画单词的罕见程度。

一般而言,我们对文档频率做一些调整:

$$idf_t = \log_{10}{N \over df_t}$$

\subsection{tf-idf权重计算与处理}

根据上面两个结果来说的话,我们可以对两个进行综合:tf用来刻画文档
对词项的相关度,而idf则用来刻画词项的稀有度。它的值如下:

$$w_{t,d} = (1+\log tf_{t,d})\cdot\log{N\over df_t}$$

由式子容易知道,tf-idf权重有性质:1. 随着词频增加而增加,2. 随着
词项罕见度增大而增大。

如同之前所说的一样,我们需要对文档形成的向量进行排序。

我们把查询也看成向量之后,就可以用余弦值来刻画它们的相近程度。在这
个之前,因为我们处理前会处理引索,如果我们把它处理为单位向量,就会大大
减少查询时间。这是因为归一化向量的积就是它们之间的余弦值。

