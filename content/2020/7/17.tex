
\section{背包问题}

背包问题是一个很基础的,又很典型的动态规划问题。

\subsection{0-1背包问题}

0-1背包问题是最简单的问题,关键在于要不然取要不然不取。

\begin{Exple}{示例}
有一个载重量为 \verb|W| 的背包,和 \verb|N| 个物品,每一个物品有重量和价值
两种属性,分别为 \verb|a[i].wt| 和 \verb|a[i].val|。

试求用这个背包装物品最大能装的价值是多少。
\end{Exple}
\begin{Exple}{解答}
一开始可能会带入到盗贼的视角,想要装下价值密度最高的,但是虽然这一般而言解
为优秀,但是实际上不一定是最优解。

这个时候只能使用动态规划:
\begin{lstlisting}
func solve(a []thing, int W) int {
    // W: 包包的容量。
    // result[i]: 如果容量为 i 的包包的最优解。
    // 注解: 一开始的时候都是 0,所以说 len(a) == 0 的情况下 result[W] == 0
    //       也很正常。不过如果初始化为 -INF,那么明显 result[index - v.wt]
    //       + v.val 的最坏情况不再是 v.val 而是 -INF 了。
    result := [W + 1]int{}
    for _, v := range a {
        for index := W; index >= v.wt; index -- {
            result[index] = max(result[index - v.wt] + v.val, result[index])
        }
    }
    return result[W]
}
\end{lstlisting}
\end{Exple}

动态规划的效率其实很低,它很大一部分都依赖于暴力解,只是使用了DP table来进行适
量的优化而已。但是在有的时候,动态规划的确就是得到最优解的唯一办法。

当然也有一些很难的0-1背包问题,比如:

\begin{Exple}{示例\footnote{更多信息请参见:\url{https://leetcode-cn.com/proble%
ms/partition-equal-subset-sum/}。}}
给定一个只包含正整数的非空数组。是否可以将这个数组分割成两个子集,使得两个子集
的元素和相等。

\impt{注意}

每个数组中的元素不会超过 100

数组的大小不会超过 200
\end{Exple}
\begin{Exple}{解答}
这道题和之前的题目的不同之处就在于:它强调了必须要把 0-1 背包填满到$sum \over 2$,
而不能是其他的价值。

在背包九讲中指出,这个和一般的 0-1 背包算法不同之处仅仅在于初始化。因为如果指定了
\verb|result[$1..W$] == -INF| 的话,那么这个时候所有的不满的结构的值都会是 %
\verb|-INF|。或者我们可以使用一个 \verb|bool| 值(我感觉这个会更好实现一些哦)。
\begin{lstlisting}
func solve(nums []int) bool {
    sum := 0
    for _, v := range nums {
        sum += v
    }
    if sum % 2 != 0 {
        return false
    }

    result := make([]bool, sum / 2 + 1)
    result[0] = true
    for int i := 1; i <= sum / 2; i ++ {
        result[i] = false
    }

    for _, v := range nums {
        for i := sum / 2; i >= v; i -- {
            result[i] = (result[i] || result[i - v])
        }
    }
    return result[sum / 2]
}
\end{lstlisting}

其实这么想的话,这样子的效率还会高那么一点点。
\end{Exple}

\subsection{完全背包问题}

完全背包的不同之处在于所有物品都是无限的。对于有的时候我们可以有一个简单的优化,
如果一个物体的被分解后数的物体的价值密度高,那么就可以把前者给直接淘汰。

\subsection{其他}

最近在看背包九章。看完了应该还会做一篇笔记吧。

%%%%%%%%%%%%%%%%%%%%%%%%%%%%%%%%%%%%%%%%%%%%%%%%%%%%%%%%%%%%%%%%%%%%%%%%%%%%%%%

\section{单调栈}

单调栈的本质还是栈,只不过它利用了一些逻辑使得在改变它本身的时候还保持有序。

主要是:在末尾排队的时候,把所有小于它的都给吃掉,这样子整个栈就是一个单调递减
的栈了,还可以保证开头的一定是最大的。

单调栈对于滑动切片或者寻找下一个最值均有良好的性能。

%%%%%%%%%%%%%%%%%%%%%%%%%%%%%%%%%%%%%%%%%%%%%%%%%%%%%%%%%%%%%%%%%%%%%%%%%%%%%%%

\section{堆}

既然说到了栈,自然也谈谈堆。堆是一种能够较好的实现取出最大/最小值的数据结构,它
的性质是:父节点始终大于/小于子节点。

基于这个,我们可以很轻松地实现堆的插入和删除:

第一,对于插入而言,我们首先把它放在尾部,然后将其整理为堆,如果该数字大于/小于
它的父节点的时候,很明显他就不是一个很好的堆,那么我们只需要将它和父节点进行调
换。
\begin{lstlisting}
func append_h(h []int, num int) {
    ind := h.size()
    h := append(h, num)
    
    for true {
        const parent_ind = (ind - 1 / 2)
        if ind != 0 && h[ind] < h[parent_ind] {
            h[ind], h[parent_ind] = h[parent_ind], h[ind]
            ind = parent_ind
        } else {
            break
        }
    }
}
\end{lstlisting}

第二,对于删除而言,我们一般指的是删除根节点。为了维持堆的稳定,我们将尾部元素
当作一个新的根节点。之后我们让它和子节点中\emph{较小的}那一个进行替换。
\begin{lstlisting}
func pop_h(h []int) (p int) {
    ind := h.size() - 1
    p, h[0] = h[0], h[ind]

    ind = 0
    for true {
        child_ind := ind * 2 + 1
        if h[child_ind] > h[child_ind + 1] {
            child_ind ++
        }
        if h[ind] > h[child_ind] {
            h[ind], h[child_ind], ind = h[child_ind], h[ind], child_ind
        } else {
            break
        }
    }
    return
}
\end{lstlisting}


