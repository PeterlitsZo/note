
\section{Fibonacci通项解}

在解出它的通项解之前,我们有它的递推公式:
$$F_i=\begin{cases}
    F_{i-1} + F_{i-2}   & \*if |i \ge 3, \\
    1                   & \*if |i \le 2.
\end{cases}$$
通过它的通项解我们能发现在一定程度上,$F_i\approx 2F_{i-1}$,那我们可以假设有$
F_i = a^i$,即等比数列,故我们可以列式:
$$a^i = a^{i-1} + a^{i-2}$$
二元一次方程解得:
$$a={1\pm \sqrt{5} \over 2}$$
但是很明显$a^1\ne a^2 \ne 1$,所以我们不妨令
$$\begin{cases}\*a|={1+\sqrt{5} \over 2} \\
\*b|={1-\sqrt{5} \over 2}\end{cases}$$
因为$\*a|^n=\*a|^{n-1}+\*a|^{n-2}$,$\*b|^n=\*b|^{n-1}+\*b|^{n-2}$,自然有:
$$\begin{cases}
    c\cdot\*a|^n + d\cdot\*b|^n = (c\cdot\*a|^{n-1} + d\cdot \*b|^{n-1}) + 
                                  (c\cdot\*a|^{n-1} + d\cdot \*b|^{n-1})    \\
    c\cdot\*a| + d\cdot\*b| = 1                                             \\
    c\cdot\*a|^2 + d\cdot\*b|^2 = 1
\end{cases}$$
那么解得:
$$\begin{cases}c = {1\over\sqrt5} \\
d = -{1\over\sqrt5}\end{cases}$$

综上所述,我们有Fibonacci的通项解:
$$F_n = c\cdot\*a|^n + d\cdot\*b|^n =
        { \left({1+\sqrt5\over2}\right)^n -
          \left({1-\sqrt5\over2}\right)^n \over\sqrt5}$$

%%%%%%%%%%%%%%%%%%%%%%%%%%%%%%%%%%%%%%%%%%%%%%%%%%%%%%%%%%%%%%%%%%%%%%%%%%%%%%%

\section{取模的艺术}

在处理一些大数的时候,总是需要对答案进行取模。

\subsection{逆元}



%%%%%%%%%%%%%%%%%%%%%%%%%%%%%%%%%%%%%%%%%%%%%%%%%%%%%%%%%%%%%%%%%%%%%%%%%%%%%%%

\section{快速幂}

在求幂的时候,有的时候我们需要尽可能快,比如说我们需要求解$A^B$,那么最简单的办
法就是:
\begin{lstlisting}
func pow(a, b int) {
    result := 1
    for i := 0; i < b; i ++ {
        result *= a
    }
    return result
}
\end{lstlisting}

注意包括上面这个例子和本节内容全部围绕着正整数集展开。上面这个例子的复杂度为$O(
b)$。因为我们得知$A^B=A^C\times A^{B-C}$,通过这个我们可以分解子任务并且进行备忘
:
\begin{lstlisting}
func pow(a, b int, M map[int]int) {
    if _, ok := M[b]; !ok {
        if b == 0 {
            M[b] = 1
        } else if b == 1 {
            M[b] = a
        }
        M[b] = pow(a, b / 2, M) * pow(a, b - b / 2, M)
    }
    return M[b]
}
\end{lstlisting}

在上面我们知道,如果$b=4$,那么只需要计算一次$A^1$和一次$A^2$就行了,但是如果$b
=5$,它则需要计算$A^1$,$A^2$和$A^3$。我们希望他能尽量采取子问题乘以子问题,也就
是说我们希望$\lfloor\frac b2\rfloor=b-\lfloor\frac b2\rfloor$,这样子我们就能构
建两个相同的子问题,从而节约算力。比如说我们希望它能够令$A^5=(A^2)^2\times A$。
\begin{lstlisting}
func pow(a, b int) {
    table := make(map[int]int)
    table[1] = a
    var i int
    for i = 2; i <= b; i *= 2 {
        table[i] = table[b / 2] * table[b / 2]
    }

    result := 1
    for i != 0 {
        if i < b {
            result *= table[i]
            b -= i
        } else {
            i /= 2
        }
    }
    return result
}
\end{lstlisting}

通过上面的代码,
这样子我们就令$A^B=\*pow|(A, \  b_1\cdot 2^1\times b_2\cdot 2^2\times\cdots\times
b_i\cdot 2^i)$,其中$b_i\in\{0, 1\}, i\in R$。其中我们只需要计算一系列$\{2^1, 2
^2, 2^3, \ldots, 2^i\}$即可,每一个$2^i=2\times 2^{i-1}$。综上复杂度为$O(\*lg |
B)=O(i)$。

除此,我们还能敏锐地发现$(b_1, b_2, \ldots, b_i)=\*base\_2\_tuple|(B)$,这是因
为二进制的性质。所以说我们能根据二进制而进行重写:
\begin{lstlisting}
func pow(a, b int) {
    last_pow, result := a, 1
    for b != 0 {
        if b & 1 != 0 {
            result *= last_pow
        }
        b >>= 1
        last_pow = last_pow * last_pow
    }
    return result
}
\end{lstlisting}

%%%%%%%%%%%%%%%%%%%%%%%%%%%%%%%%%%%%%%%%%%%%%%%%%%%%%%%%%%%%%%%%%%%%%%%%%%%%%%%

\section{多校\#1-1005题解}

\subsection{}

%%%%%%%%%%%%%%%%%%%%%%%%%%%%%%%%%%%%%%%%%%%%%%%%%%%%%%%%%%%%%%%%%%%%%%%%%%%%%%%

% 0. 按照p排序。
% 1. 先淘汰掉永远超不过车的车子(a_i<a_{i+1},p_{i}<p_{i+1})的。
% 2. 维护一个树,算出第i个车超过i+1个车所需要的时间点。
% 3. 对时间点最小的i而言,将i+1出栈,重新计算i到i+1(新的i+1)的时间点,其中i+1如果
% 是leader的话,result++。
% 4. 问题:对于上面那个特殊情况,如果时间点一样的话,那么result+=k?。

