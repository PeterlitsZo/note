
%%%%%%%%%%%%%%%%%%%%%%%%%%%%%%%%%%%%%%%%%%%%%%%%%%%%%%%%%%%

\section[HDU-1358题解]{HDU-1358题解\footnote{更多信息参见:\url{%
http://acm.hdu.edu.cn/showproblem.php?pid=1358}。}}

看了一看这道题我在比赛的时候搞了有那么久,但是好像都没戏。
好像是构建了一个自动机解决但是他说是TLE,这太奇怪了,毕竟
一个不放回的自动机的时间复杂度按理来说应该是$O(n)$才对呀。

\subsection{题干}

For each prefix of a given string $S$ with $N$ characters
(each character has an ASCII code between 97 and 126,
inclusive), we want to know whether the prefix is a periodic
string. That is, for each $i$ ($2 \le i \le N$) we want to
know the largest $K > 1$ (if there is one) such that the
prefix of $S$ with length $i$ can be written as $A^K$ ,
that is $A$ concatenated $K$ times, for some string $A$.
Of course, we also want to know the period $K$.

\impt{Input}
The input file consists of several test cases. Each test
case consists of two lines. The first one contains $N$
($2 \le N \le 1,000,000$) – the size of the string $S$.
The second line contains the string $S$. The input file
ends with a line, having the number zero on it.

\impt{Output}
For each test case, output “\verb|Test case #|” and the
consecutive test case number on a single line; then, for
each prefix with length $i$ that has a period $K > 1$,
output the prefix size $i$ and the period $K$ separated by
a single space; the prefix sizes must be in increasing
order. Print a blank line after each test case.

\subsection{自己的思考}

一开始看到这个题的时候,就马上构建了一个有限状态自动机,
很明显之后就错掉了啦。

想了一想可能的确和字符串算法有关吧?

我的想法如下:
\begin{lstlisting}
int $K$ = 1        // $K\ge 1$ is the loop times of $A$, initally as 1
string $A$ = ""    // the loop basic part
for index, character in $S$:
    // if all thing is at the state of inital
    if $A$ == "":
        $A$ += character
    // if the index-th character match the $A$
    elif character == $S$[index % $A$.len]:
        if (index + 1) % $A$.len == 0:
            $K$ += 1
    // if the index-th do not match $A$
    else:
        $A$ = $S$[:index]
        $K$ = 1

    // output
    // ...
\end{lstlisting}

(附注:后来重新上传了一次,结果他们都是用的KMP算法来构建next数组。
我不是很喜欢,而且也不会,更重要的是,我不知道我到底错在哪里了
啦。唉。真的搞不懂。)

(附注:之后是一次WA的submit,问了问同学,对于循环节 “\verb|aaba|”
之类的来说,就会有判断错误,换言之,对于proper suffix和proper perfix相
同的循环节来说,这个算法判断不了在什么时候断开循环节。它会在开始的时候
把proper suffix和proper perfix错误匹配,如果匹配过多的话就会错过断开循
环节的时机。而历史的车轮是不允许开倒车的。下面还是看看KMP和如何构造KMP的
partial match table)

\subsection[KMP算法原理]{KMP算法原理\footnote{更多信息参见:\url{%
http://jakeboxer.com/blog/2009/12/13/the-knuth-morris-pra%
tt-algorithm-in-my-own-words/}。}}

题解说这个涉及到了KMP算法。所以我们应该先了解一下所谓的KMP算法
到底是什么。

KMP算法是一种字符匹配算法。它和依次比较的算法不同,它充分使用了
匹配中的额外信息进行更好的匹配。

直观性解释。KMP算法是一种字符匹配算法,比较而言,最差的情况是遍历
然后逐字符比较,所以说,真正nice的方法是尽量少比较(每个最多匹配
一次),能跳就跳,在这个基础上我们可以构建一个辅助数组能满足下
列操作:
\begin{lstlisting}
abccbaabacabababababca
||)                      // match 2 characters -> skip (2 - 0) = 2
abababca

abccbaabacabababababca
xx....|||)               // match 3 characters -> skip (3 - 1) = 2
      abababca

abccbaabacabababababca
      xx=)               // match 1 characters -> skip (1 - 0) = 1
        abababca

abccbaabacabababababca
        x.||||)          // match 4 characters -> skip (4 - 2) = 2
          abababca

abccbaabacabababababca
          xx==||||)      // match 2 characters -> skip (6 - 4) = 2
            abababca

abccbaabacabababababca
            xx====||||   // OK
              abababca
\end{lstlisting}

很明显,如果是原来的那种糟糕的比较方式的话,那么对$n$的文章和$m<n$的搜索字符串
的复杂度为$O(mn)$,而采用KMP比较方法的话,则复杂度为$O(m+n)$。

\def\ttstr{\text{\verb|string|}}

观察可知,对于 $\ttstr = [a_1, a_2, \ldots, a_i, \ldots, a_n]$而言,
有(相应而言和)后缀$[a_i, \ldots, a_{i+k}] = $前缀$[a_1, \ldots, 
a_{1+k}]$,而且在匹配过程中这些位数匹配,但是接下来的
那一个位数不匹配的时候,那么我们可以向前移位$i-1$位\footnote{为什么?
:因为移位之后,前$1+k$位仍然匹配。很明显前一项是proper suffix,而后一项
是proper perfix,这两个相等。}然后接着继续上一个匹配的地方继续匹配。

最重要的就是建表,只要的核心是前缀和后缀是否匹配。为了这个patrial match table。
建表的意义就是让最大的proper suffix和proper prefix匹配对用上合理的地方。

举个例子,proper suffix就是字符串中为前缀但不是字符串本身的字符串,比如对于“$abcd$”
的前缀就有“$abc$”,“$ab$”,“$a$”。同理它的proper suffix应该是“$bcd$”,“$cd$”,“$d$”。
如果这个时候,已经匹配过的地方中的有一个proper suffix和proper perfix最长匹配对,而下
一个又没有匹配上的话,那么自然让他们这个匹配对重合,然后接着匹配就vans了。自然跳
过 \verb|matched.len - suffix.len| 个单位就没事了。

我感觉讲明白KMP算法很困难。但是我感觉我应该是理解了吧。我在这里只说了一下
为什么可以得到这个匹配的位置,但是没有谈到为什么第一个匹配的就一定是它。

\subsection{KMP算法的next数组}

目前我们知道了next数组表达的是什么\footnote{即,最长的匹配对的长度。}%
。但是我们还不知道如何能够快速构建next数组。而快速构建next数组正是KMP算法
的核心\footnote{我的天!这么厉害的吗!是核心吗?}。

我们先预先构建一张典型的表如表\ref{KMP next table}。

\begin{table}[h]
    \centering
    \begin{tabular}{l|llllllll}
        \toprule
            index & 0 & 1 & 2 & 3 & 4 & 5 & 6 & 7 \\
        \midrule
            char  & a & b & a & b & a & b & c & a \\
            value & 0 & 0 & 1 & 2 & 3 & 4 & 0 & 1 \\
        \bottomrule
    \end{tabular}
    \caption{KMP算法的next数组示例}
    \label{KMP next table}
\end{table}

对$index$为4的地方寻找它的$value$为例,我们可以看到一个非常具有代表性的
PMT构建实例:我们假设已经构建出了前面$index$为$0\sim 3$的字符的$value$,
很明显对于$index$为3的字符而言:存在长度为2的匹配前后缀:“\verb|ab|”。
我们依次向后面多看一位,有情况“\verb|abx|”和“\verb|aby|”。我们可以看到
\verb|x|$=$\verb|y|$=$\verb|a|,那么对于$index=4$的前缀字符串而言,他有
匹配的字符串 \verb|aba|$=$\verb|aba|。那么此时有$value_4=value_3+1$。

从上面的一个实例而言我们有:$value_n=value_{n-1}+1$,当$char_{
value_{n-1}}=char_n$的时候(这个实例中,正是因为$char_2=char_4$,所以
$value_4=value_3+1$)。

和这个实例不一样的是:如果下一个字符串 \verb|x|$\ne$\verb|y| 了该怎么办?
看向index为6的字符“\verb|c|”,在它前面的字符“\verb|b|”根据它的对应value,
我们知道了$S[0\sim5]$字符串的匹配对为长度为4的“\verb|abab|”,但是它这个
前缀的下一位是“\verb|a|”,和index为6的“\verb|c|”严重不符。我们可以这么想:
$S[0\sim3]=S[2\sim5]=\text{\verb|abab|}$,而我接下来的任务是在这个前后
匹配的字符串中寻找一个更小的字符串,有$S[0\sim x]+S[x+1]=S[5-x\sim5]+
\text{“\verb|c|”}$。值得注意的是,由于匹配对的性质,所以它们是完全一样的,
既有:$S[5-x\sim5]=S[3-x\sim3]$,所以最快速的做法是在$S[0\sim3]$中先预先
快速寻找到最长的匹配字符串,而这恰恰是之前已经做过的 --- 长度为2的字符串
“\verb|ab|”,可惜$S[3]\ne \text{\verb|c|}$,所以接下来的任务就是寻找“\verb|ab|”
中的子串,然后依次递归解体。

\subsection{题解}

这道题主要是判断周期性字符串。那么已知了next数组的情况下,怎么才可以判断
是否前面和本身一起构建了一个周期性字符串呢?我们可以从最长前缀后缀匹配对
来入手。就像奇函数和偶函数一样,很容易就可以知道,当字符长度$len$和最长匹配
前后缀对的长度$len'$满足$len'\mod (len-len') = 0$的情况下,它是一个周期性字
符串。

所以这道题的主流做法就是构建next数组+寻找合适的位置。

Python代码:
\begin{lstlisting}
def get_next(string):
    # inital state
    next = [0 for i in range(len(string))]
    ok = False

    # calc the next[i, from 1 to string.len]
    for i in range(1, len(string)):
        # firstly, the inital temp is the character before self.
        temp = i - 1

        # then if self do not match the prefix's next character, then
        # smallen variable `temp` as the substring's end index.
        while string[i] != string[next[temp]]:
            if temp != 0:
                temp = next[temp] - 1
            else:
                next[i] = 0
                ok = True
                break

        # if the variable `temp` point the begin, we need to be sure
        # if self really match 0-index character.
        if not ok:
            next[i] = next[temp] + 1

    return temp
\end{lstlisting}

当然这个非常符合逻辑,但是每一次都需要重新计算一次 \verb|temp| 的值,
所以有一个通用的计算方式,用来固定住它,获得更高的效率:
\begin{lstlisting}
def get_next(string):
    next = [0 for i in range(len(string))]
    self = 1
    now = 0

    while self < len(string):
        if string[self] == string[now]:
            now += 1
            next[self] = now
            self += 1
        elif now:
            now = next[now - 1]
        else:
            # now == 0
            self += 1
\end{lstlisting}

其中的变量 \verb|now| 既是next数组中对应的下标,也是 \verb|self| 对应
的最长的匹配对的长度。

\subsection[扩展]{扩展\footnote{更多信息参见:\url{http://%
acm.hdu.edu.cn/showproblem.php?pid=3336}。}}

It is well known that AekdyCoin is good at string problems
as well as number theory problems. When given $a$ string $s$,
we can write down all the non-empty prefixes of this string.
For example:
\begin{lstlisting}
s: "abab"
The prefixes are: "a", "ab", "aba", "abab"
\end{lstlisting}

For each prefix, we can count the times it matches in $s$.
So we can see that prefix \verb|"a"| matches twice, \verb|"ab"|
matches twice too, \verb|"aba"| matches once, and \verb|"abab"|
matches once. Now you are asked to calculate the sum of the
match times for all the prefixes. For \verb|"abab"|, it is $2
+ 2 + 1 + 1 = 6$.

The answer may be very large, so output the answer $\mod 10007$.

\impt{Input}
The first line is a single integer $T$, indicating the number of
test cases.

For each case, the first line is an integer $n$ ($1 \le n \le 200000$),
which is the length of string $s$. A line follows giving the
string $s$. The characters in the strings are all lower-case letters.

%%%%%%%%%%%%%%%%%%%%%%%%%%%%%%%%%%%%%%%%%%%%%%%%%%%%%%%%%%%

\section{HDU-1690题解}

%%%%%%%%%%%%%%%%%%%%%%%%%%%%%%%%%%%%%%%%%%%%%%%%%%%%%%%%%%%

\section{CF-1041-E题解}

%%%%%%%%%%%%%%%%%%%%%%%%%%%%%%%%%%%%%%%%%%%%%%%%%%%%%%%%%%%

\section{CF-1023-D题解}

%%%%%%%%%%%%%%%%%%%%%%%%%%%%%%%%%%%%%%%%%%%%%%%%%%%%%%%%%%%

\section{CF-1017-D题解}

%%%%%%%%%%%%%%%%%%%%%%%%%%%%%%%%%%%%%%%%%%%%%%%%%%%%%%%%%%%

\section{CF-1206-D题解}

%%%%%%%%%%%%%%%%%%%%%%%%%%%%%%%%%%%%%%%%%%%%%%%%%%%%%%%%%%%

