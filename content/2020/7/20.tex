
\section[LeetCode-32题解]{LeetCode-32题解\footnote{更多情况请参见:\url{https:/%
/leetcode-cn.com/problems/longest-valid-parentheses/}。}}

这是一个关于括号的题目,一开始我在想使用线段树来做,结果括号不具备最优子结构。我
因为这个甚至直接排除了动态规划的做法,很明显我错了,换一个角度它就有好看的动态规
划转移方程了。

\subsection{题干}

题目如下:
\begin{Exple}{最长有效括号}
给定一个只包含 \verb|'('| 和 \verb|')'| 的字符串,找出最长的包含有效括号的子串的长度。

示例1:
\begin{lstlisting}
输入: "(()"
输出: 2
解释: 最长有效括号子串为 "()"
\end{lstlisting}

示例2:
\begin{lstlisting}
输入: ")()())"
输出: 4
解释: 最长有效括号子串为 "()()"
\end{lstlisting}
\end{Exple}

\subsection{我的想法}

在无法使用线段树的时候(因为 \verb|(((| 和 \verb|)))| 不是有效括号,但是它们加起
来又的的确确变成了一个有效括号,换言之,这是括号序列不具备可加性,所以应该构建不
出来线段树)。

在这么想之后,我开始尝试使用自动机来判别有效括号。
\begin{lstlisting}
func getMaximumSubString(s string, index int) int {
    cnt, back_up_index := 0, index
    for index < len(s) {
        switch s[index] {
        case ')': cnt --
        default:  cnt ++
        }
        if cnt == 0 {
            back_up_index = index + 1
        } else if cnt < 0 {
            return index
        }
        index ++
    }
    return back_up_index
}
\end{lstlisting}

这个自动机会在找到以 \verb|index| 开头的最长字符串之后就返回结束位置的后一位。
因为我们可以断言两个长字符串是不可能在一起的,因为在一起之后它们一定是一个有效
字符串了。所以不存在一个长有效字符串也在另一个长有效字符串中。

\subsection{动态规划题解}

我们定义一个数组$dp$,其中$dp_i$是以$i$下标结尾的最长有效字符串长度(注意,有最
优子结构的是长度而不是字符串,意思是长度可以通过前面的推算出来),而且我们有状态
转移方程:$$dp_i = \begin{cases}
    dp_{i-2} + 2                    & \T if | s[0:i+1]\T like|\t'.*()'| \\
    dp_{i-1} + dp_{i-dp[i-1]-2} + 2 & \T if | s[0:i+1]\T like|\t'.*))'|
                                      \T and | s_{i-dp[i-1]-1}=\t'('|
\end{cases}$$
其中,$dp_{i-dp[i-1]-2}$是$\t(|s[i-dp[i-1]:i-1]\t)|$前面的有效字符长度。

%%%%%%%%%%%%%%%%%%%%%%%%%%%%%%%%%%%%%%%%%%%%%%%%%%%%%%%%%%%%%%%%%%%%%%%%%%%%%%%

