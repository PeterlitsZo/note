
\section{线段树}

\begin{epi}
    \Text{线段树是算法竞赛中常用来维护区间信息的数据结构。}
    \Source{OI wiki}
\end{epi}

\subsection{引入和细节}

在子序列的最大和中有例子:
\begin{Exple}{示例}
给定一个整数数组$nums$,找到一个具有最大和的连续子数组(子数组最少包含一个元素),返回其最大和。

\impt{示例}
比如说有以下测试:
\begin{lstlisting}
input:   [-2,1,-3,4,-1,2,1,-5,4]
output:  6
explain: 6 = sum([4, -1, 2, 1])
\end{lstlisting}
\end{Exple}

首先关于一维DP是一个方法,但是还有另外一种比较复杂但是适用面广的方法:即线段树。
定义一个函数$get(a, l, r)$,它的返回值就是在$[l, r]$的最大和的连续子数组,为了
通过分治的方法得到它,我们使用四个属性来定义线段树的节点:
\begin{enumerate}
\item \verb|lSum| 用来表示以 \verb|l| 为开始端点的最大字段和。
\item \verb|rSum| 用来表示以 \verb|r| 为结束端点的最大子段和。
\item \verb|mSum| 用来表示它本身的最大子段和。
\item \verb|iSum| 用来表示它本身的区间和。
\end{enumerate}

那么我们同时就可以很轻松地明白如何维护高阶的四个状态,或者说如何自底向上:
\begin{enumerate}
\item \verb|lSum| 可以是左区间的 \verb|lSum| 或者是左区间的 \verb|iSum| 再加上右
区间的 \verb|lSum|。(如果右区间的 \verb|lSum| 过于庞大,以至于比左区间的 %
\verb|iSum - lSum| 还要大)。
\item \verb|rSum| 和 \verb|lSum| 基本上同理。
\item \verb|mSum| 比较复杂一些,可能是左区间的 \verb|mSum| 或者是右区间的 %
\verb|mSum| 甚至是 \verb|rSum + lSum|。
\item \verb|iSum| 就是左右区间的 \verb|iSum| 相加。
\end{enumerate}

最关键的是,如果我们将其用堆的方式进行储存的话,那么我们也可以在后续中多次查询,
同时复杂度也不是很高。(复习一下,如果堆的根节点为$d_1$,那么对于节点$d_i$而言,
它的左节点为$d_{2i}$而右节点为$d_{2i+1}$)。

%%%%%%%%%%%%%%%%%%%%%%%%%%%%%%%%%%%%%%%%%%%%%%%%%%%%%%%%%%%%%%%%%%%%%%%%%%%%%%%

\section{子序列的最大和}

子序列的最大和,即所有子序列的和的最优解的值。

\subsection{引入与题解}

首先是一个最简单的LeetCode的题目\footnote{更多信息请参见:\url{https://leetcode%
-cn.com/problems/maximum-subarray/}。}:
\begin{Exple}{示例}
给定一个整数数组$nums$,找到一个具有最大和的连续子数组(子数组最少包含一个元素),返回其最大和。

\impt{示例}
比如说有以下测试:
\begin{lstlisting}
input:   [-2,1,-3,4,-1,2,1,-5,4]
output:  6
explain: 6 = sum([4, -1, 2, 1])
\end{lstlisting}
\end{Exple}
\begin{Exple}{题解}
首先这种题目的话,一开始呢会想到暴力求解,比方说:
\begin{lstlisting}
func solve(nums []int) int {
    const L := len(nums)
    result := -INF
    for len := 1; len <= L; len++ {
        for index := 0; index <= L - len; index++ {
            result = max(result, sum(nums[index:index + len]))
        }
    }
    return result
}
\end{lstlisting}

但是这种解法不能是最优解,它的复杂度甚至高达$O(\t L|^2)$。

那该怎么办呢?
\end{Exple}

一开始我想使用DP,因为这不是一个值而是一个序列所以我感觉应该使用二维DP来做!其实
不是的,这个题不能用二维DP,用了的话和暴力解的复杂度都差不多了!这个其实有一个非
常简单的动态规划转移方程:
$$f(n) = max\{f(n-1)+a_n, a_n\}$$
这样我们就定义了一个\emph{必须以$a_n$结尾的子序列序列的解}的最大值了。我们只需要
在状态转移的途中记录下最大值就万事大吉了。而这种恰恰好好可以定义一个完整的子序列
而不是分散的子序列。(如果是分散的话,那么的确应该用多维DP了吧。)如代码:
\begin{lstlisting}
func solve(nums []int) int {
    max := -INF
    result := 0
    for _, v := range nums {
        if (result > 0) {
            result += v
        } else {
            result = v
        }
        if result > max {
            max = result
        }
    }
    return max
}
\end{lstlisting}

%%%%%%%%%%%%%%%%%%%%%%%%%%%%%%%%%%%%%%%%%%%%%%%%%%%%%%%%%%%%%%%%%%%%%%%%%%%%%%%

\section{LeetCode-312题解}

\subsection{题干}

\impt{戳气球}

有$n$个气球,编号为$0$到$n-1$,每个气球上都标有一个数字,这些数字存在数组$nums$中。

现在要求你戳破所有的气球。如果你戳破气球$i$,就可以获得$nums[left]*nums[i]*nums[right]$个硬币。这里的$left$和$right$代表和$i$相邻的两个气球的序号。注意当你戳破了气球$i$后,气球$left$和气球$right$就变成了相邻的气球。

求所能获得硬币的最大数量。

\impt{说明:}

你可以假设$nums[-1]=nums[n]=1$,但注意它们不是真实存在的所以并不能被戳破。

$0\le n\le 500, 0\le nums[i]\le 100$

\subsection{我的想法}

我看着很像是DP的题目但是又下笔写不出来。一般我遇到这种题目的时候总是想要自顶向
下,但是有时候会比较麻烦,尤其是状态涉及到一个array或者tuple的时候,传值必须要
使用指针将一个vector或者slice传入过去。

但是如果采用自底向上的话,最不济也就是一个多维数组,而且容易将重复子问题进行记
忆化。我们还可以进行抽象,比如说我要填满整个数组,并不需要知道它的所有子问题的
气球群状态,我们可以把子问题也抽象成填满子区间(这种抽象就更容易建模了)。

比如对于这道题而言我可以这样写:
\begin{lstlisting}
function maxCoins(nums []int) {
    // initally
    n := len(nums)
    rec := make([][]int, n + 2)
    for i := 0; i < n + 2; i ++ {
        rec[i] = make([]int, n + 2)
    }
    val := make([]int, n + 2)
    for i := 1; i <= n; i ++ {
        val[i] = nums[i - 1]
    }
    val[0], val[n + 1] = 1, 1

    // main part
    for i := n - 1; i >= 0; i -- {
        for j := i + 2; j <= n + 1; j ++ {
            for k := i + 1; k < j; k++ {
                sum := val[i] * val[k] * val[j]
                sum += rec[i][k] + rec[k][j]
                rec[i][j] = max(rec[i][j], sum)
            }
        }
    }
    return rec[0][n+1]
\end{lstlisting}

最关键的地方在于递归式:%
\verb|sum = val[i] * val[k] * val[j] + rec[i][k] + rec[k][j]|。意思是,在
\verb|i| 和 \verb|j| 中间插入一个下标为 \verb|k| 的值,然后在填充这两侧。其求得
的值,即为 \verb|max(...)| 函数的一部分。

%%%%%%%%%%%%%%%%%%%%%%%%%%%%%%%%%%%%%%%%%%%%%%%%%%%%%%%%%%%%%%%%%%%%%%%%%%%%%%%

