%%%%%%%%%%%%%%%%%%%%%%%%%%%%%%%%%%%%%%%%%%%%%%%%%%%%%%%%%%%%%%%%%%%%%%%%%%%%%%%

\section{6月笔记}

%%%%%%%%%%%%%%%%%%%%%%%%%%%%%%%%%%%%%%%%%%%%%%%%%%%%%%%%%%%%%%%%%%%%%%%%%%%%%%%

\subsection{To-do List}

\begin{plttodoenv}{3}
\t[ ]重新做一下HDU-6400
\end{plttodoenv}

%%%%%%%%%%%%%%%%%%%%%%%%%%%%%%%%%%%%%%%%%%%%%%%%%%%%%%%%%%%%%%%%%%%%%%%%%%%%%%%

\section{Lua\LaTeX{}的使用}

今天为了解决计算时间太慢的问题,找了一下Lua\LaTeX{}的tutorial,发现的确不错,
对于我来说,只需要改动一点点地方就可以运行,能够完美支持\vb|ctexart|的底层就是
一个好底层。

而我最喜欢的就是:它本身可以运行Lua代码,而这个代码的速度是非常非常快的,作为
一个脚本语言做到这么快还是不容易,而且标准库的算法很不错。(我就是因为,原来
的宏包把时间转换为具体的时间很慢才找到Lua\LaTeX{}的,我翻译原来的那个,是每一
次都从头加到尾,还要判断闰年什么的,不建表)。

其次,宏展开是优先于运行代码的,所以很容易很容易就可以向Lua传入参数,输出就用%
\vb|luadreact|(好像不是这么写的),输出到原文件中,就像用\vb|JavaScript|来操纵%
\vb|HTML|一样。太棒了啦。

%%%%%%%%%%%%%%%%%%%%%%%%%%%%%%%%%%%%%%%%%%%%%%%%%%%%%%%%%%%%%%%%%%%%%%%%%%%%%%%

\section{\LaTeX{}的长度}

最近,在写\LaTeX{}宏包的时候,出现了一点幽灵长度,我总是搞不清楚,为什么高一点
的和矮一点的间距不平等,后来我终于搞懂了,记录在案。

\subsection{段落、文字行之间的间距}

一般来说,两个段落之间是由宏\vb|\par|而隔开的,所以说那两个段落之间的竖直盒子
是由\vb|\par|命令决定的,而\vb|\par|以变量,长度寄存器变量\vb|\parskip|来决定,
它的值是\vb|0pt plus 1pt|,弹性长度下限一定是固定的,但是上限不是固定了(当然
如果超过上限太多了,那它就是一个坏盒子了)。

文字行之间的间距,这是由\vb|\baselineskip|决定了。这就是说,如果我的这一行盒子
都差不多是一个字符高的话,那么它乖乖地采用下限,留出好看的间隙,如果超过了,那
它的高度就不是上限了,而是它本身的高度了。

\subsection{解决方案}

定义\vb|\baselineskip|为\vb|0pt|,这样的话,无论是多高的话,它的高度就是它本身
高度了,而不会留间隙,固定的间隙应该用\vb|vspace*|来代替它,用在两个命令之间,
构建一个漂亮的间隙。

%%%%%%%%%%%%%%%%%%%%%%%%%%%%%%%%%%%%%%%%%%%%%%%%%%%%%%%%%%%%%%%%%%%%%%%%%%%%%%%

\section{HDU-6400题解}

杭州电大的ACM官网好慢哦,注册一下又要等半天。
然后好像官网的题解也没有,有点难受。

\subsection{题目}

这道题,说的是有一个括号矩阵,给定了$n$和$m$作为
长和高,然后让行匹配和列匹配的个数最多。
那么什么情况下是匹配的呢:一个正向括号就是匹配的,
反之则不匹配。

比如:
\begin{lstlisting}
1.          2.              3.
(()())      ((()(())))      (
                            )
                            (
                            )
\end{lstlisting}

这三个分别有行匹配和列匹配的情况。计数为1。我们
要对于给定$n$和$m$而言选定计数最大的进行输出。
很明显,这个不是唯一的。

\subsection{想法}

在做这个之前我有一点想法,但是做错了。

首先对于正向括号序列而言,那么有:它的长度$len$一定满足:%
$$len (\text{mod } 2) = 0$$

所以说如果这个是一个奇数(odd number)的话,无论如何也不可能是一个正向括号序列。
所以有:

\begin{enumerate}
\item $n$和$m$都是奇数,那么计数$cnt$一定是零。
\item $n$是奇数但是$m$不是,那么计数$cnt$为$n$(因为$n$对应的那一个构造成正向括号
的话,计数$cnt$就为$n$,这个时候是最优解)。反之亦然。
\item 我感觉最难的\vb|>_<|,那就是$n$和$m$都是偶数。这种情况下我想了一个绝妙的想法
那就是,令$a=max(n, m)$,$b=min(n,m)$,那么有如果我把$a$对应的行/列都改成正向括号
序列,那么答案计数$cnt$就是$a$了。但是不够,后来我有想到:
\begin{lstlisting}
(()) => (())
()() => ()  ()
(()) => (())
()() => ()  ()
\end{lstlisting}

像这种的话,就太棒了!首先它本身有四个正向序列构成。还见缝插针的搞了一个竖的(第二列
)所以说,只要我在原有基础上多插插就有多的了。这个时候,计数$cnt$的值就为%
$a+\frac{b}{2}-1$,其中$\frac{b}{2}$是因为原第一排只有一半是开括号,可以用来
构建正向序列,减一是因为第一个开括号是不能搞事情的。

\subsection{题解}

只能说我想到了一点但是没有想到第二点。这道题需要构建正向括号序列,但是其实
正向的它的本质是开头是\vb|(|末尾是\vb|)|然后其他的只要保证中间不跌下到-1(
正向序列的值,意思是有没有匹配上的\vb|)|反括号)并且末尾为零就好了。

那么这么看的话,
\begin{lstlisting}
.((((((. => .((((((.
(......) => ()()()()
(......) => (()()())
(......) => ()()()()
(......) => (()()())
.)))))). => .)))))).
\end{lstlisting}

上面的才是最优解,只要它中间那一坨能够满足的话整个的计数就是$cnt=a+b-4$,
两个式子比较有\begin{align}&(a+b-4)-(a+\frac{b}{2}-1)\\&=b-\frac{b}{2}-3\\
&=\frac{b}{2}-3\\&>0\\&\Longrightarrow b>6\end{align},所以说在大一点的时候
应该用这个方法会比较好一点。

\end{enumerate}
