\section{Python的编码问题}
\timetx{2020-01-03}

尝试用\code{print('...', file=out\_put\_file)}来进行文本输出,
发现输出的文本不是用\code{UTF-8}来编码的,
而是好像用的国标(国家的标准?)

在\code{str}中字符是用\code{unicode}来编码的,是没有被\code{encode}的二进制数据,
输出到文件,输出的内容编码是由系统的配置决定的。

猜想:\code{open}时会指定\code{encoding}可以解决这个问题。

\code{open}时指定\code{encoding}会有带有编码信息的属性的文本对象。
因为有状态的影响,所以该对象的\code{write method}会用编码下的二进制实现去写文件。

而\code{print}可以指定\code{file},但是它(应该)不会读取文件内部的编码状态,
(这个时候我想起来了,有时用\code{u8}编码的时候好像会在文件头留下一个特别的标记?)
而是直接用系统指定的编码格式去输出二进制数据去写数据。

猜想:\code{print}是把文件输出到\code{sys.stdout}上(应该,至少意思是这样)。
输出的环境是由系统的配置/环境决定的,为了环境能够正常显示文本数据,
就应该会用配置下的文件编码格式来编码\code{Unicode}的数据流,
这也是为什么它不会去读取文本文件的编码状态的原因吧?
(甚至还没有参数可以调节输出的编码状态,这样)
